\section*{Assignment 1: Platform Concept and Value Proposition}
\addcontentsline{toc}{section}{Assignment 1: Platform Concept and Value Proposition}

The platform developed during this course, \textbf{SkillSync}, connects students hunting for real projects with small organisations that need help but lack budgets for consultants. Our core promise is a scoped collaboration that gives students portfolio wins while NGOs unlock motivated talent for a few intense weeks.

The concept only clicked after several messy iterations. Early notes just said ``students helping real-world actors,'' which classmates rightly called a slogan. Guided by \citet{Choudary2016} and \citet{Srnicek2017} we tightened the idea into a project-based platform that sits between internships and gig work: flexible enough to dodge HR red tape yet structured enough to deliver measurable outcomes.

SkillSync therefore behaves as a two-sided orchestrator. Students bring skills and energy; NGOs and civic teams bring real problems. We obsess over trustworthy matchmaking so cross-side network effects can blossom. Institutional email verification, lightweight vetting, and a templated scoping wizard keep expectations aligned before anyone spends serious time.

Data is the quiet engine. Every completed project generates structured feedback, endorsements, and behavioural signals. In the short run we refine matching and keep quality high; in the longer run we craft portable skill passports, giving us an edge in the credentialing space \citet{Zuboff2019} critiques. The value proposition lives in those feedback loops as much as in the pitch deck.

Figure~\ref{fig:student-view} shows how the student-facing dashboard keeps that promise tangible. The left column surfaces curated projects that match skill tags, while the right column holds nudges drawn from mentoring rituals. Project cards highlight stipend range, time commitment, and expected deliverables because uncertainty on those points kills motivation. The ``mentor check-in'' strip pulls data from earlier conversations so guidance stays personal rather than generic.

\begin{figure}[h]
  \centering
  \includegraphics[width=0.85\linewidth]{projektvisning-student.png}
  \caption{Student project view (`projektvisning-student.png`) that operationalises the SkillSync value proposition.}
  \label{fig:student-view}
\end{figure}

On the supply side we created a mirrored experience for NGOs. The creation wizard walks through a scoping checklist in plain language: desired outcome, must-have skills, support on offer. We tested the template with our two anchor NGOs until they could complete it in under eight minutes. The figure in Assignment~3 shows that workflow in action. The value proposition is not only a pitch line about ``students meet projects''; it is a set of micro-interactions that reduce uncertainty for both sides and make repeat usage more likely.

To wrap up, the English-language, expanded version of Assignment~1 shows how SkillSync translates course concepts into practice. We centre one core interaction, leverage network effects, keep marginal costs tiny, and treat fairness as a strategic asset rather than afterthought. The platform is not a generic marketplace; it is a carefully orchestrated arena where early-career value gets produced through trust-rich collaboration, and the newly elaborated details make that ambition tangible. The figure tour and expanded process notes lock those ideas into observable artefacts instead of abstract bullet points.
