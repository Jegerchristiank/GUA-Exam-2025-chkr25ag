\section*{Assignment 09: Scaling Strategy}
\addcontentsline{toc}{section}{Assignment 09: Scaling Strategy}

SkillSync scales in three stages. Phase one tackles product-market fit: fortify the core, test network effects within one geographic cluster, and land two local anchor partners (trade association plus municipal innovation unit) so their signalling power lifts both sides at once \citep{Choudary2016,Reillier2017}. Success looks like 60 active students, 20 projects completed with satisfaction above 4.5/5, and partner retention over 70\%. We keep a focused growth pod, two developers for stability, and a community manager shepherding feedback loops in the pilot Slack.

Phase two covers regional scaling. We standardise onboarding flows and API contracts so partners plug in without hand-holding, run a three-tier programme (community, certified, strategic) to manage quality and incentives \citep{HagiuWright2013}, staff a lean partner-success team, ship shared dashboards, and track cross-side conversion plus time-to-value per partner to see whether network effects accelerate \citep{ShapiroVarian1999,Lecture12}. Targets include 250 active students, 75 active organisations, and time-to-first-value under 12 hours.

\begin{figure}[H]
  \centering
  \includegraphics[width=0.75\linewidth]{Student-Dashboard.png}
  \caption{Adoption dashboard tracking activation against partner enablement.}
  \label{fig:scaling-dashboard}
\end{figure}

Figure~\ref{fig:scaling-dashboard} ties activation gains directly to partner readiness.

Phase three moves national (maybe Nordic) once the first two phases prove unit economics. We court alliances with larger institutional players, negotiate white-label deals with select enterprise clients, and expand governance with clear data-sharing principles, algorithm audits, and an advisory board so legitimacy scales \citep{Srnicek2017,Zuboff2019}. The threshold is 500 projects per year with completion above 90\% and net revenue retention above 110\%. Budget covers compliance expertise, localisation, and a small deal desk that can tailor partnerships without wrecking the standard product.

Two risks dominate: churn and quality decay. Churn can hit users or partners, especially if competitors tempt them with exclusive features or lower fees, so we build switching costs through data portability, loyalty loops, and analytics that lose value if someone leaves \citep{FarrellSaloner1986,ShapiroVarian1999}. Quality decay flares when growth dilutes standards, so we enforce service-level agreements, automate match-quality monitoring, and run quarterly partner reviews with a community board catching soft signals before the dashboards scream \citep{Reillier2017}. Example safeguards include limiting access to longitudinal impact reports to active partners and running spot audits on project retros.

Theory lines up with the platform canon: network effects need critical mass but pushing too fast erodes match quality and differentiation \citep{Porter2008}. \citet{Choudary2016} remind us governance and producer enablement must evolve with each phase, while \citet{Srnicek2017} stresses pairing growth with legitimacy and transparency, so we keep investing in partnerships and organisational scaffolding.

We stress-tested the roadmap with a simple Google Sheets simulation covering activation rate, partner velocity, moderation load, and average project value. It tells us how many moderators and partner managers we need per 1,000 active users, how big a churn buffer to hold, and when to pause expansion: if quality drops below 4.4/5 for two months we freeze new partners until the governance board approves remediation.

Each phase ends with a ``story harvest'' that feeds marketing, training, and retros via \citet{Choudary2016}'s interaction-model canvas.
