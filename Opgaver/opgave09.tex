\section*{Assignment 09: Scaling Strategy}
\addcontentsline{toc}{section}{Assignment 09: Scaling Strategy}

SkillSync scales in three stages. Phase one tackles product-market fit: fortify the core, test network effects within one geographic cluster, and land two local anchor partners (trade association plus municipal innovation unit) so their signalling power lifts both sides at once \citep{Choudary2016,Reillier2017}. Success looks like 60 active students, 20 projects completed with satisfaction above 4.5/5, and partner retention over 70\%. I keep a focused growth pod, two developers for stability, and a community manager shepherding feedback loops in the pilot Slack --- all still hypothetical roles for now.

Phase two covers regional scaling. I would standardise onboarding flows and API contracts so partners plug in without hand-holding, run a three-tier programme (community, certified, strategic) to manage quality and incentives \citep{HagiuWright2013}, staff a lean partner-success team, ship shared dashboards, and track cross-side conversion plus time-to-value per partner to see whether network effects accelerate \citep{ShapiroVarian1999,Lecture12}. Targets include 250 active students, 75 active organisations, and time-to-first-value under 12 hours.

\begin{figure}[H]
  \centering
  \includegraphics[width=0.75\linewidth]{figures/Student-Dashboard.png}
  \caption{Adoption dashboard mock-up tracking activation against partner enablement.}
  \label{fig:scaling-dashboard}
\end{figure}

Phase three moves national (maybe Nordic) once the first two phases prove unit economics. The roadmap courts alliances with larger institutional players, negotiates white-label deals with select enterprise clients, and expands governance with clear data-sharing principles, algorithm audits, and an advisory board so legitimacy scales \citep{Srnicek2017,Zuboff2019}. The threshold is 500 projects per year with completion above 90\% and net revenue retention above 110\%.

Two risks dominate: churn and quality decay. Churn can hit users or partners, especially if competitors tempt them with exclusive features or lower fees, so the plan builds switching costs through data portability, loyalty loops, and analytics that lose value if someone leaves \citep{FarrellSaloner1986,ShapiroVarian1999}. Quality decay flares when growth dilutes standards, so I would enforce service-level agreements, automate match-quality monitoring, and run quarterly partner reviews with a community board catching signals before the dashboards scream \citep{Reillier2017}. Example safeguards include limiting access to longitudinal impact reports to active partners and running spot audits on project retros.

Theory lines up with the platform canon: network effects need critical mass but pushing too fast erodes match quality and differentiation \citep{Porter2008}. \citet{Choudary2016} remind us governance and producer enablement must evolve with each phase, while \citet{Srnicek2017} stresses pairing growth with legitimacy and transparency, so I keep investing in partnerships and organisational scaffolding on paper.

I stress-tested the roadmap with a simple Google Sheets simulation covering activation rate, partner velocity, moderation load, and average project value. It estimates moderators and partner managers per 1,000 active users, a churn buffer, and a pause rule: if quality drops below 4.4/5 for two months I freeze new partners until the governance board approves remediation.

Each phase in the plan ends with a ``story harvest'' that would feed marketing, training, and retros via \citet{Choudary2016}'s interaction-model canvas.
