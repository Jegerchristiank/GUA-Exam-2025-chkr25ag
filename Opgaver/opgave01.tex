\section*{Assignment 01: Platform Concept and Value Proposition}
\addcontentsline{toc}{section}{Assignment 01: Platform Concept and Value Proposition}

The platform we built, \textbf{SkillSync}, connects students chasing real projects with small organisations that need help but lack consultant budgets. The core promise is a scoped collaboration that gives students portfolio wins while NGOs unlock motivated talent for a few intense weeks.

The concept only clicked after several messy iterations. Early notes just said ``students helping real-world actors,'' which classmates rightly called a slogan. Guided by \citet{Choudary2016} and \citet{Srnicek2017} we tightened the idea into a project-based platform that sits between internships and gig work: flexible enough to dodge HR red tape yet structured enough to deliver measurable outcomes.

SkillSync therefore behaves as a two-sided orchestrator. Students bring skills and energy; NGOs and civic teams bring real problems. We obsess over trustworthy matchmaking so cross-side network effects can blossom. Institutional email verification, lightweight vetting, and a templated scoping wizard keep expectations aligned before anyone spends serious time, exactly the launch hygiene hammered home in Lecture~2 on platform network effects \citep{Lecture02}.

Data is the quiet engine. Every project generates feedback, endorsements, and behavioural signals. In the short run we refine matching and keep quality high; in the longer run we craft portable skill passports, giving us an edge in the credentialing space \citet{Zuboff2019} critiques. The value proposition lives inside those loops as much as in the pitch deck.

Figure~\ref{fig:student-view} keeps the promise tangible. The refreshed `Student-Project-View.png` layout highlights curated projects while pinning progress nudges and mentor notes on the right. Project cards emphasise stipend range, time commitment, and deliverables because uncertainty kills motivation, and the ``mentor check-in'' strip pulls past data so guidance stays personal.

\begin{figure}[H]
  \centering
  \includegraphics[width=0.85\linewidth]{Student-Project-View.png}
  \caption{Student project view (`Student-Project-View.png`).}
  \label{fig:student-view}
\end{figure}

On the supply side we created a mirrored experience for NGOs. The creation wizard walks through a scoping checklist in plain language: desired outcome, must-have skills, support on offer. We tested the template with our two anchor NGOs until they could complete it in under eight minutes. The figure in Assignment~3 shows that workflow in action. The value proposition is not only a pitch line about ``students meet projects''; it is a set of micro-interactions that reduce uncertainty for both sides and make repeat usage more likely.

To wrap up, this expanded Assignment~1 shows how SkillSync translates course concepts into practice. We centre one core interaction, leverage network effects, keep marginal costs tiny, and treat fairness as a strategic asset rather than an afterthought. The platform is not a generic marketplace; it is a carefully orchestrated arena where early-career value emerges through trust-rich collaboration, and the figure tour plus process notes anchor that ambition in observable artefacts.
