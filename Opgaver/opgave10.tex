\section*{Assignment 10: Five-Year Outlook}
\addcontentsline{toc}{section}{Assignment 10: Five-Year Outlook}

\subsection*{Scenario building process}
I built the five year outlook using a conservative base case plus sensitivity checks. The base case assumes SkillSync launches in one city, scales regionally by year three, and reaches national coverage by year five. Growth rates draw from Assignment~09's targets, and retention reflects the benchmarks compiled from \citet{Choudary2016}, \citet{Srnicek2017}, and Lecture~13 \citep{Lecture13}. Desk research on comparable civic platforms suggests conversion rates around twelve percent, so I used that as the starting point.

\subsection*{Projected outcomes}
The model produces three headline numbers:
\begin{itemize}
  \item \textbf{Active users.} Seventy five thousand active participants by year five, assuming fifty five percent annual growth after the regional phase and retention stabilising at sixty eight percent. If retention slips five points, active users fall to fifty eight thousand.
  \item \textbf{Revenue.} Forty two million DKK derived from the seven percent completion fee, enablement subscriptions, and insight reports. Gross margin reaches fifty eight percent once onboarding automation reduces support costs. Delaying automation by twelve months cuts revenue by roughly nine million DKK.
  \item \textbf{Strategic partnerships.} Eighteen formal agreements: three national anchor institutions, five sector data hubs, and ten municipal innovation units. Each agreement requires a governance memorandum that mirrors Assignment~05's data and fairness commitments.
\end{itemize}

\subsection*{Threats and responses}
Major risks include substitution, multi homing, and regulatory shifts. If a global marketplace drops prices, SkillSync deploys a ten percent discount fund financed by grants and doubles down on transparent impact reporting that generic rivals lack \citep{Porter2008}. To counter multi homing, the platform keeps investing in integrations and longitudinal analytics that lose value if exported, echoing \citet{FarrellSaloner1986}. Regulatory pressure is managed through proactive audits, public transparency reports, and rapid response playbooks as suggested by \citet{Srnicek2017}.

I also considered exit scenarios. In a downside case SkillSync pivots into a data infrastructure service that powers university project portals. Another scenario involves an acqui hire by a larger civic platform, which demands clean contracts, modular code, and escrowed backups. Documenting these options applies \citet{Reillier2017}'s advice to plan for platform evolution even when things go well.

\subsection*{Readiness rituals}
To keep the outlook alive, I designed an annual "readiness week" ritual. Day one revisits network effect diagnostics from Lecture~1 and Lecture~2. Day two reviews monetisation and metrics from Lecture~5. Day three focuses on governance and inequality using material from Lecture~10 and Lecture~11. Day four hosts alumni led red team drills inspired by Lecture~12. Day five concludes with a go or no go memo following Lecture~13's wrap up prompts \citep{Lecture01,Lecture02,Lecture05,Lecture10,Lecture11,Lecture12,Lecture13}. This cadence ensures strategy stays grounded in theory rather than wishful thinking.

\subsection*{Closing reflection}
Balancing ambition and responsibility remains the hardest part. The course readings keep me honest: \citet{Choudary2016} pushes me to optimise the core interaction, \citet{Porter2008} stresses differentiation, and \citet{Srnicek2017} demands legitimacy. If SkillSync keeps those voices at the table, the five year horizon looks achievable.
