\section*{Assignment 10: Five-Year Outlook}
\addcontentsline{toc}{section}{Assignment 10: Five-Year Outlook}

This outlook maintains analytical discipline regarding ambition and risk while pulling together the wrap-up prompts from Lecture~13 \citep{Lecture13}.

\subsection*{Projections}
I mapped a conservative base scenario with three headline numbers, building on the groundwork from earlier assignments:
\begin{itemize}
  \item \textbf{Active users:} 75,000, assuming 55\% annual growth from local cluster launches and retention at 68\% via proactive partner enablement \citep{Choudary2016,Srnicek2017}.
  \item \textbf{Revenue:} 42 million DKK from 7\% transaction fees, data-informed subscriptions, and co-branded partnerships with 58\% gross margin once support is automated \citep{ShapiroVarian1999}.
  \item \textbf{Strategic partnerships:} 18 deals (three national anchors, five sector data hubs, ten municipal or regional innovation units) under governance memoranda \citep{Reillier2017}.
\end{itemize}

The numbers stem from pilot data (conversion around 12\%) and the assumption that by years two and three we automate onboarding so partners can integrate without bespoke development. Sensitivity checks maintain analytical discipline: a five-point drop in retention shrinks the active-user number to 58,000, while delaying automation by a year cuts revenue by roughly 9 million DKK. If viral loops hit harder, usage and revenue might land 30\% higher, but only if community features resonate.

\subsection*{Threats and exit scenarios}
The major threats remain the classics from platform literature: substitution, multi-homing, and regulation. If a global player dumps prices our fees suddenly look expensive \citep{Porter2008}, so we keep a 10\% discount fund, co-marketing agreements, and impact reporting that generic marketplaces rarely match. Transparency failures could let regulators choke data flows \citep{Srnicek2017}, so we publish governance updates, pre-draft responses, and rehearse through tabletop exercises. Multi-homing stays persistent, which is why we protect the integrations and longitudinal insights that make switching costly \citep{FarrellSaloner1986}.

I sketch two realistic exit options if everything collapses: either a controlled acqui-hire where a larger Nordic platform buys the team and IP while we sunset the marketplace, or a pivot into pure data infrastructure that maintains the API layer as SaaS for a smaller client base. Both options demand modular code, clean contracts, quarterly documentation weeks, and escrowed backups \citep{Reillier2017}.

Before we lock the five-year plan, I sketched a ``readiness week'' ritual each June: revisit platform fundamentals and network-effect diagnostics \citep{Lecture01,Lecture02}, rerun monetisation and governance tests \citep{Lecture05,Lecture10}, review inequality and data-ethics metrics with stakeholders \citep{Lecture08,Lecture11}, stage alumni-led red teams \citep{Lecture12}, and close with a go/no-go memo in the spirit of Lecture~13 \citep{Lecture13}. The ritual keeps the outlook a living hypothesis and doubles as onboarding for new hires.

\subsection*{Closing reflection}
This journey is a reminder of how demanding it is to balance growth ambitions with governance. Every design choice hits both sides of the marketplace simultaneously, forcing loop thinking rather than linear funnels \citep{Choudary2016}. Porter’s competitive strategy stays a reality check on differentiation \citep{Porter2008}, and network effects only compound when legitimacy, data quality, and partnerships stay front and centre \citep{Srnicek2017}.

By year five, I expect SkillSync to operate as a civic infrastructure layer where universities plug skill gaps, NGOs find an innovation sandbox, and students see a rite of passage, but only if we keep investing in transparency and monitor Assignment~08's metrics.

