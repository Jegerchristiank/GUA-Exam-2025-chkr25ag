\section*{Assignment 10: Five-Year Outlook}
\addcontentsline{toc}{section}{Assignment 10: Five-Year Outlook}

\subsection*{Scenario building process}
I built the five year outlook using a careful base case and extra sensitivity checks. The base case assumes SkillSync starts in one city, grows to nearby regions by year three, and reaches national coverage by year five. Growth rates come from Assignment~09’s targets, and retention follows the benchmarks from \citet{Choudary2016}, \citet{Srnicek2017}, and Lecture~13 \citep{Lecture13}. Desk research on similar civic platforms shows conversion rates of about twelve percent, so I used that as the starting point.

\subsection*{Projected outcomes}
The model produces three headline numbers:
\begin{itemize}

\item \textbf{Active users.} 75,000 active participants by year five, assuming 55 percent annual growth after the regional phase and retention stabilising at 68 percent. If retention slips five points, active users fall to 58,000.

\item \textbf{Revenue.} 42 million DKK derived from the 7 percent completion fee, enablement subscriptions, and insight reports. Gross margin reaches 58 percent once onboarding automation reduces support costs. Delaying automation by 12 months cuts revenue by roughly 9 million DKK.

\item \textbf{Strategic partnerships.} 18 formal agreements: three national anchor institutions, five sector data hubs, and ten municipal innovation units. Each agreement requires a governance memorandum that mirrors Assignment~05's data and fairness commitments.

\end{itemize}

\subsection*{Threats and responses}
Major risks include \textit{substitution}, \textit{multi homing}, and regulatory shifts. If a global marketplace lowers prices, SkillSync uses a ten percent discount fund supported by grants and strengthens its transparent impact reporting that generic competitors lack \citep{Porter2008}. To handle \textit{multi homing}, the platform keeps improving its integrations and long-term analytics that lose value if exported, following \citet{FarrellSaloner1986}. Regulatory pressure is managed through regular audits, public transparency reports, and quick response plans as described by \citet{Srnicek2017}.

I also considered exit scenarios. In a downside case, SkillSync could shift into a \textit{data infrastructure service} that supports university project portals. Another possibility is an \textit{acqui hire} by a larger \textit{civic platform}, which requires clear contracts, modular code, and backups. Describing these options follows \citet{Reillier2017}'s advice to plan for \textit{platform evolution} even when progress is positive.

\subsection*{Readiness rituals}
To keep the outlook alive, I designed an annual "readiness week" ritual. Day one revisits network effect diagnostics from Lecture~1 and Lecture~2. Day two reviews monetisation and metrics from Lecture~5. Day three focuses on governance and inequality using material from Lecture~10 and Lecture~11. Day four hosts alumni led red team drills inspired by Lecture~12. Day five concludes with a go or no go memo following Lecture~13's wrap up prompts \citep{Lecture01,Lecture02,Lecture05,Lecture10,Lecture11,Lecture12,Lecture13}. This cadence ensures strategy stays grounded in theory rather than wishful thinking.

\subsection*{Closing reflection}
Balancing ambition and responsibility remains the hardest part. The course readings keep me honest: \citet{Choudary2016} pushes me to optimise the core interaction, \citet{Porter2008} stresses differentiation, and \citet{Srnicek2017} demands legitimacy. If SkillSync keeps those voices at the table, the five year horizon looks achievable.
