\section*{Assignment 06: Competitive Positioning}
\addcontentsline{toc}{section}{Assignment 06: Competitive Positioning}

\subsection*{Landscape and pressure points}
To avoid falling in love with my own idea I mapped the ecosystem using Porter’s five forces \citep{Porter2008}. Direct competitors include Worksome, LinkedIn’s project marketplace, and student-consulting collectives already partnering with NGOs. Substitutes abound: organisations can hire interns, tap volunteer portals, or lean on pro bono consultancies, while students can multi-home on Upwork, hackathons, or extracurricular societies. Supplier power shows up in universities that can revoke access to student communities, so the strategy leans on co-created curricula and data-sharing agreements. Buyer power is meaningful because NGOs operate on tight budgets, so pricing stays transparent and ties fees to completed value. Threat of new entrants stays high given low technical barriers, making differentiation live in the interaction I choreograph rather than defensive contracts. Multi-homing risk remains the critical pressure point.

\subsection*{Moats I can realistically build}
Platform theory reminds me that sustainable advantage comes from reinforcing loops rather than traditional lock-in \citep{Choudary2016,Reillier2017,Lecture07}. I see three pillars:
\begin{enumerate}
  \item \textbf{Trust-rich matching.} Every project goes through a scoping template and a student--NGO advisory circle, keeping quality high and lowering perceived risk so generic freelance sites struggle to poach matches.
  \item \textbf{Data-powered enablement.} Analytics translate project outcomes into skills passports for students and impact dashboards for NGOs, and the history we capture makes nuanced matchmaking harder to copy \citep{FarrellSaloner1986}.
  \item \textbf{Community partnerships.} We embed with campus career centres and municipal innovation labs that already broker collaborations, creating distribution moats because they trust us with their communities \citep{ShapiroVarian1999}.
\end{enumerate}

\subsection*{Strategic moves}
To operationalise the pillars I sketch three moves: launch a ``project assurance'' programme where we co-pilot the first sprint for new NGOs so reliability and case studies build fast; ship an open API that lets universities sync SkillSync activity into their learning systems, raising switching costs without being anti-competitive; and publish quarterly transparency reports on outcomes, diversity, and data usage to reinforce the ethical stance from Assignment~05 and position the platform as a legitimacy-first player \citep{Srnicek2017,Zuboff2019}. The assurance programme doubles as buyer-power mitigation because it lowers perceived risk for NGOs that could otherwise negotiate discounts.

To back the strategy I benchmarked SkillSync against four archetypes: global freelance marketplaces, local student consultancies, university project portals, and specialised volunteer networks. SkillSync wins on curated governance (fast dispute resolution and co-piloted onboarding), ties on project breadth, and deliberately loses on raw scale because the concept prioritises trust over volume. The comparison exposed a messaging gap, which nudged me to design the chat system in Assignment~07 and reinforced that platform advantage is a systems problem, not a single feature.

A ``red team'' exercise in my notes had me role-play upstart competitors. One scenario imagined a global tech company cloning the concept but subsidising NGOs with cloud credits, countered by civic partnerships and proprietary quality data; another pictured universities rebuilding in-house, which I offset with rapid experimentation and analytics insights NGOs would rate highly. Documenting the drills maintains strategic discipline by keeping investment focused on relational assets that make copying hard.
