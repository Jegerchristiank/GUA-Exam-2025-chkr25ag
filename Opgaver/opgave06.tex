\section*{Assignment 06: Competitive Positioning}
\addcontentsline{toc}{section}{Assignment 06: Competitive Positioning}

\subsection*{Porter analysis with explicit evidence}
I ran Porter’s five forces on a shared spreadsheet and logged assumptions. Competitive rivalry is high because Worksome, LinkedIn projects, and student consulting clubs already serve parts of the market. I documented their pricing, onboarding friction, and trust signals during desk research. Threat of substitutes is significant: NGOs can hire interns, use volunteer portals, or stick with pro bono consultants, while students can join hackathons or freelance elsewhere. Supplier power rests with universities that control access to student communities and with mentors who lend credibility. Buyer power stems from NGO budget constraints, so pricing must correlate with delivered outcomes. Threat of new entrants stays elevated because the technical barrier is low. This breakdown mirrors \citet{Porter2008}'s template and helped prioritise which edges matter.

\subsection*{Differentiation pillars tied to theory}
Rather than chase artificial lock in, I anchor advantage in reinforcing loops as advised by \citet{Choudary2016} and \citet{Reillier2017}.
\begin{enumerate}
  \item \textbf{Trust heavy matching.} Each project runs through the scoping wizard, mentor review, and kickoff ritual. These steps reduce uncertainty and align with \citet{HagiuWright2013}'s reminder that platforms win when both sides trust the match quality.
  \item \textbf{Learning rich data.} Project reflections feed skill passports for students and impact dashboards for NGOs. Over time, this history becomes a unique dataset that improves matching and partner decisions, echoing \citet{FarrellSaloner1986}'s insight on switching costs derived from information advantages.
  \item \textbf{Institutional partnerships.} By co designing curricula and reporting with universities and municipal labs, SkillSync embeds within existing governance structures. \citet{ShapiroVarian1999} note that distribution advantages matter as much as product features, so these alliances serve as a durable moat.
\end{enumerate}

\subsection*{Strategic moves and validation}
To activate those pillars I plan three moves. First, launch a project assurance programme where the SkillSync team co pilots the first sprint for any new organisation. Success is defined as a satisfaction score above four point five and a testimonial collected within two weeks. Second, ship an open API so universities can sync project data into learning systems, making multi homing less attractive without blocking it. Third, publish quarterly transparency reports covering outcomes, diversity, and data usage. This move extends the governance promises from Assignment~05 and reflects \citet{Srnicek2017}'s demand for legitimacy through openness.

I pressure tested the strategy by building a competitor matrix comparing SkillSync to freelance marketplaces, student consultancies, university portals, and volunteer networks. SkillSync scores highest on curated governance and trust rituals but lowest on sheer scale. That trade off is intentional: Lecture~7 reminded us that platform advantage stems from superior interactions, not only volume \citep{Lecture07}. I also ran a red team exercise imagining two attack scenarios. In one, a global tech player clones the product but subsidises fees with cloud credits. The counter involves deepening civic partnerships and protecting the analytics insights that require our governance process. In the other, universities attempt to rebuild in house. The defence is faster experimentation and joint ownership of data so that leaving would mean losing validated impact records. Documenting these drills keeps the strategy evidence based.
