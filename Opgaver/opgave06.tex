\section*{Assignment 06: Competitive Positioning}
\addcontentsline{toc}{section}{Assignment 06: Competitive Positioning}

\subsection*{Porter analysis with explicit evidence}

I used \textit{Porter’s five forces} in a shared spreadsheet and wrote down the main assumptions. \textit{Competitive rivalry} is high because Worksome, LinkedIn projects, and student consulting clubs already cover parts of the market. I noted their pricing, onboarding challenges, and trust signals during desk research. The \textit{threat of substitutes} is also strong since NGOs can hire interns, use volunteer portals, or continue with pro bono consultants, while students can join hackathons or freelance elsewhere. \textit{Supplier power} lies with universities that control access to student communities and with mentors who add credibility. \textit{Buyer power} comes from NGO budget limits, so pricing must match real outcomes. The \textit{threat of new entrants} stays high because the technical barrier is low. This overview follows \citet{Porter2008}'s framework and helped identify which factors matter most.

\subsection*{Differentiation pillars tied to theory}
Rather than chase artificial lock in, I anchor advantage in reinforcing loops as advised by \citet{Choudary2016} and \citet{Reillier2017}.
\begin{enumerate}
  \item \textbf{Trust heavy matching.} Each project runs through the scoping wizard, mentor review, and kickoff ritual. These steps reduce uncertainty and align with \citet{HagiuWright2013}'s reminder that platforms win when both sides trust the match quality.
  \item \textbf{Learning rich data.} Project reflections feed skill passports for students and impact dashboards for NGOs. Over time, this history becomes a unique dataset that improves matching and partner decisions, echoing \citet{FarrellSaloner1986}'s insight on switching costs derived from information advantages.
  \item \textbf{Institutional partnerships.} By co designing curricula and reporting with universities and municipal labs, SkillSync embeds within existing governance structures. \citet{ShapiroVarian1999} note that distribution advantages matter as much as product features, so these alliances serve as a durable moat.
\end{enumerate}

\subsection*{Strategic moves and validation}
To activate those pillars I plan three moves. First, I will launch a project assurance programme where the SkillSync team co pilots the first sprint for every new organisation. Success means getting a satisfaction score above four point five and a testimonial within two weeks. Second, I will ship an \textit{open API} so universities can sync project data with learning systems, making \textit{multi homing} less attractive without blocking it. Third, I will publish quarterly transparency reports about outcomes, diversity, and data usage. This step continues the governance promises from Assignment~05 and supports \citet{Srnicek2017}'s call for legitimacy through openness.

I tested the strategy by building a competitor matrix that compared SkillSync to freelance marketplaces, student consultancies, university portals, and volunteer networks. SkillSync scored highest on \textit{curated governance} and \textit{trust rituals} but lowest on overall scale. This difference was intentional. Lecture~7 explained that a \textit{platform advantage} depends on better interactions, not just higher volume \citep{Lecture07}.  

I also ran a \textit{red team exercise} with two possible attack scenarios. In the first, a global tech company copies the product and reduces prices using cloud credits. The response focuses on creating stronger \textit{civic partnerships} and protecting the \textit{analytics insights} that depend on our governance process. In the second, universities try to build their own version. The defence focuses on faster experimentation and shared \textit{data ownership}, so leaving the platform would mean losing verified impact records. Recording these drills keeps the strategy based on evidence.