\section*{Assignment 06: Competitive Positioning}
\addcontentsline{toc}{section}{Assignment 06: Competitive Positioning}

\subsection*{Landscape and pressure points}
To avoid falling in love with our own idea I mapped the ecosystem using Porter’s five forces \citep{Porter2008}. Direct competitors include Worksome, LinkedIn’s project marketplace, and student-consulting collectives already partnering with NGOs. Substitutes abound: organisations can hire interns, tap volunteer portals like VolunteerMatch, or lean on pro bono consultancies, while students can multi-home on Upwork, hackathons, or extracurricular societies. Multi-homing risk is sky high, so differentiation has to live in the interaction we choreograph, not defensive contracts.

\subsection*{Moats we can realistically build}
Platform theory reminds us that sustainable advantage comes from reinforcing loops rather than traditional lock-in \citep{Choudary2016,Reillier2017,Lecture07}. I see three pillars:
\begin{enumerate}
  \item \textbf{Trust-rich matching.} Every project goes through a scoping template and is reviewed by a student--NGO advisory circle. That keeps quality high and lowers perceived risk for both sides, making it harder for generic freelance sites to poach our best matches.
  \item \textbf{Data-powered enablement.} We invest in analytics that translate project outcomes into skills passports for students and impact dashboards for NGOs. The more history we capture, the harder it becomes for rivals to replicate the nuanced matchmaking without years of data, echoing \citet{FarrellSaloner1986}'s take on compatibility advantages.
  \item \textbf{Community partnerships.} We embed with campus career centres and municipal innovation labs that already broker collaborations. Those partnerships act as distribution moats because they trust us with their communities, a softer barrier \citet{ShapiroVarian1999} say often trumps hard technology advantages.
\end{enumerate}

\subsection*{Strategic moves}
To operationalise the pillars we commit to three moves: launch a ``project assurance'' programme where we co-pilot the first sprint for new NGOs so reliability and case studies build fast; ship an open API that lets universities sync SkillSync activity into their learning systems, raising switching costs without being anti-competitive; and publish quarterly transparency reports on outcomes, diversity, and data usage to reinforce the ethical stance from Assignment~05 and position us as the legitimacy-first platform \citep{Srnicek2017,Zuboff2019}.

To back the strategy we benchmarked SkillSync against four archetypes: global freelance marketplaces, local student consultancies, university project portals, and specialised volunteer networks. SkillSync wins on curated governance (fast dispute resolution and co-piloted onboarding), ties on project breadth, and deliberately loses on raw scale because we prioritise trust over volume. The comparison exposed a messaging gap, which pushed us to build the chat system in Assignment~07 and reminded us that platform advantage is a systems problem, not a single feature.

We also ran a ``red team'' exercise where classmates role-played upstart competitors. One scenario imagined a global tech company cloning us but subsidising NGOs with cloud credits, which we countered with the civic partnership network and proprietary quality data. Another pictured universities rebuilding in-house; we defended with speed of experimentation and the analytics insights NGOs rate highly. Documenting the drills keeps the strategy honest: the moats only hold if we keep investing in the relational assets that make copying hard.
