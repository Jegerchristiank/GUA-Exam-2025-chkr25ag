\documentclass[12pt,a4paper]{article}

% Encoding, language, fonts
\usepackage[T1]{fontenc}
\usepackage[utf8]{inputenc}
\usepackage[english]{babel}
\usepackage{newtxtext,newtxmath}
\usepackage{xcolor}
\usepackage[protrusion=true,expansion=true]{microtype}

% Layout and floats
\usepackage[a4paper,margin=2.5cm]{geometry}
\usepackage{setspace}
\setstretch{1.5}
\usepackage{graphicx}
\graphicspath{{figures/}{../figures/}}
\usepackage{float}
\usepackage{caption}
\captionsetup[figure]{
  font=normalsize,
  labelfont=bf,
  width=1\linewidth,
  justification=centering
}
\usepackage{fancyhdr}

% Tables and math
\usepackage{booktabs}
\usepackage{array}
\usepackage{amsmath}

% Citations (natbib-compatible)
\usepackage[natbibapa]{apacite}


% Cover page metadata
\newcommand{\university}{Copenhagen Business School}
\newcommand{\faculty}{Department of Digitalisation}
\newcommand{\examTitle}{Google, Uber, Amazon: Management of Platform Business}
\newcommand{\examCode}{BHAAV6034U.LECTURE\_E25}
\newcommand{\examType}{Take-home Exam Submission}
\newcommand{\studentName}{Christian Kristensen}
\newcommand{\studentId}{chkr25ag}
\newcommand{\submissionDate}{31 October 2025}
\newcommand{\wordCount}{29,259}

\begin{document}
\begin{titlepage}
  \thispagestyle{empty}
  \centering
  {\Large \textbf{\university}}\\[0.5cm]
  {\large \faculty}\\[1.5cm]
  {\LARGE \textbf{\examTitle}}\\[0.5cm]
  {\large \examCode\\\examType}\\[1.5cm]
  \begin{tabular}{rl}
    \textbf{Student name:} & \studentName \\
    \textbf{Student number:} & \studentId \\
    \textbf{Submission date:} & \submissionDate \\
    \textbf{Character count (incl. spaces):} & \wordCount \\
  \end{tabular}\\[1.5cm]
  \vfill
  {\large Course responsible: Associate Professor Carina Antonia Hallin}\\[0.3cm]
  {\large Academic term: Autumn 2025}\\[0.3cm]
  {\large Exam window: 17 October 2025 -- 31 October 2025}\\[1.5cm]
  \rule{0.8\linewidth}{0.4pt}\\[0.5cm]
  {\small This cover sheet confirms that the assignment is submitted individually and complies with CBS formal requirements.}
\end{titlepage}

% Remove numbering in headings and thus in-text
\setcounter{secnumdepth}{-1}
% Remove numbers in TOC entries
\makeatletter
\renewcommand{\numberline}[1]{}
\makeatother

\tableofcontents
\newpage

\section*{Introduction}
\addcontentsline{toc}{section}{Introduction}
This submission captures my two-week take-home exam reflections on the SkillSync platform we built in class. I keep the informal workshop voice but ground every answer in the course theory so an examiner can follow the logic without joining our project meetings.

The ten assignments form a guided tour: the opening questions cover value proposition plus the first network and monetisation loops, the middle ones document governance, data policy, and inequality, and the closing trio focus on metrics, scaling bets, and the five-year outlook. Each chapter leans on prototype artefacts---screens, feedback logs, and experiment notes---to keep the argument observable instead of vibe-based.

To orient the reader, Figure~\ref{fig:intro-showcase} revisits the poster we field-tested during the course fair. It sets the tone for the rest of the paper: a student-built platform solving real coordination problems for NGOs and ambitious students, analysed through the frameworks of \citet{Choudary2016}, \citet{Srnicek2017}, \citet{Reillier2017}, and the wider platform strategy canon discussed across the semester \citep{Lecture01,Lecture03,Lecture05}.

\begin{figure}[h]
  \centering
  \includegraphics[width=0.7\linewidth]{figures/Poster.png}
  \caption{Poster session snapshot (`Poster.png`).}
  \label{fig:intro-showcase}
\end{figure}

\section*{Assignment 01: Platform Concept and Value Proposition}
\addcontentsline{toc}{section}{Assignment 01: Platform Concept and Value Proposition}

The platform concept I develop, \textbf{SkillSync}, imagines a bridge between students chasing real projects and small organisations that need help but lack consultant budgets. SkillSync would act as digital infrastructure facilitating repeated exchanges between distinct user groups \citep{Choudary2016}. The core promise is a scoped collaboration that gives students portfolio wins while NGOs unlock motivated talent for a few intense weeks.

The concept only clicked after several quiet iterations. Early notes just said ``students helping real-world actors,'' which felt like a slogan more than a design. Guided by \citet{Choudary2016} and \citet{Srnicek2017} I tightened the idea into a project-based platform that sits between internships and gig work: flexible enough to dodge HR red tape yet structured enough to deliver measurable outcomes once it exists.

SkillSync therefore behaves, in theory, as a two-sided orchestrator. Students would bring skills and energy; NGOs and civic teams would bring real problems. The design emphasises trustworthy matchmaking so cross-side network effects can blossom. Institutional email verification, lightweight vetting, and a templated scoping wizard sit on the draft checklist to keep expectations aligned before anyone spends serious time, echoing the launch hygiene emphasised in Lecture~2 on platform network effects \citep{Lecture02}.

Data is the quiet engine in the imagined operating model. Each project could generate feedback, endorsements, and behavioural signals. In the short run, that data would refine matching; in the longer run, it could craft portable skill passports, nodding to the credentialing space \citet{Zuboff2019} critiques. The value proposition therefore lives inside those loops as much as in the pitch deck.

Figure~\ref{fig:student-view} renders the promise as a mock-up. The sketched student dashboard highlights curated projects while pinning progress nudges and mentor notes on the right. Project cards emphasise stipend range, time commitment, and deliverables because uncertainty kills motivation, and the ``mentor check-in'' strip imagines past data keeping guidance personal.

\begin{figure}[H]
  \centering
  \includegraphics[width=0.85\linewidth]{figures/Student-Project-View.png}
  \caption{Student project workspace mock-up with progress nudges.}
  \label{fig:student-view}
\end{figure}

On the supply side, I mirror the experience for NGOs. The creation wizard walks through a scoping checklist in plain language: desired outcome, must-have skills, support on offer. Desk research suggests such a template could be completed within minutes. The figure in Assignment~3 shows that workflow in action. The value proposition is therefore more than a slogan about ``students meet projects''; it is a set of proposed micro-interactions that reduce uncertainty for both sides and make repeat usage more likely.

\section*{Assignment 02: Network Effects and Launch Strategy}
\addcontentsline{toc}{section}{Assignment 02: Network Effects and Launch Strategy}

\subsection*{Network effects mapped explicitly}
I followed \citet{Choudary2016}'s three step loop mapping exercise to keep the network effects honest. Step one \textbf{defines the core interaction}: a vetted organisation posts a scoped brief, curated students respond, and both sides commit to a sprint. Step two \textbf{specifies reinforcing signals}. The cross side loop grows when visible success stories and quick responses reassure the next wave of users. Student same side effects stem from peer endorsements and a ritual where alumni leave short Loom videos once a sprint ends, echoing Lecture~4's point that \textit{social proof reduces cold start friction} \citep{Lecture04}. Organisation same side effects emerge when NGOs see comparable peers succeeding, so I pencilled monthly showcase calls. Finally \textbf{the data loop records skills used}, hours invested, and satisfaction scores, upgrading the matching algorithm each cycle and nudging SkillSync toward the orchestrator pattern \citep{Reillier2017}.

\subsection*{Solving the penguin problem}
To break the mutual hesitation \citet{HagiuWright2013} warn about, the launch plan recruits two anchor NGOs already comfortable mentoring students. They become the first proof points and agree to share testimonials. Parallel to that, I recruit about 40 students through faculty recommendations, student societies, and the course Slack so quality stays predictable. A fast onboarding script walks both sides through the first mission: Manually review briefs, pair mentors, and host a kickoff call. Students receive travel stipends for the first sprint funded by a faculty innovation grants, while NGOs get a temporary fee waiver that expires after two successful projects, following \citet{FarrellSaloner1986}'s logic on \textit{introductory pricing}.

Communication routines are scripted in advance. Weekly check-ins and a shared calendar keep the first ten projects on track; we log every difficult point and feed it into a FAQ public for all. With inspiration from Lecture~4 guidance: \textit{when seeding a platform, design commitment devices that keep early adopters active long enough for loops to form} \citep{Lecture04}. I also plan a backup: if a project fails mid sprint, SkillSync deploys a standby student pair from the founding 40 students, within forty eight hours, so organisations trust us despite hiccups.

\subsection*{Launch measurements and decision rules}
Because there is no live \textit{product} yet, my reflections read \textit{pre mortem}. I focus on \textbf{three metrics}. Value measures \textbf{minutes from signup to first action}; if it exceeds 30 minutes the team must simplify the process. \textbf{Match completion rate}, tracks how many pairs finish within scope; a dip below 70\% triggers a session (With root cause). Early net promoter score captures qualitative trust; scores below plus twenty prompt follow up interviews. These numbers come from \citet{ShapiroVarian1999}'s advice to \textit{anchor monetisation on observed value} and from Lecture~5's insistence that early experiments should move one behaviour at a time \citep{Lecture05}. Figure~\ref{fig:application-flow} visualises the guided student flow that underpins these metrics.

\begin{figure}[H]
  \centering
  \includegraphics[width=0.85\linewidth]{figures/Student-Submission.png}
  \caption{Student application flow mock up with guided pitch submission.}
  \label{fig:application-flow}
\end{figure}

I mirror the journey for organisations in Assignment~03. The idea is to choreograph both sides simultaneously so that once the early subsidies phase out, the playbook already embeds trust instead of improvised fixes.

\section*{Assignment 03: Evolution of the Platform Concept}
\addcontentsline{toc}{section}{Assignment 03: Evolution of the Platform Concept}

\subsection*{Why the pivot happened}
My earliest sketches centred on a dinner experience marketplace. Applying \citet{Choudary2016}'s asset light checklist exposed two conflicts: the concept required controlling physical spaces and guaranteeing food safety, which would turn the platform into an operator rather than an orchestrator. Lecture~6 also highlighted how platforms that ignore regulatory friction misread winner takes all dynamics \citep{Lecture06}. Combining those lessons with \citet{Srnicek2017}'s critique of extractive gig models convinced me to stop investing energy there.

\subsection*{Using the platform design toolkit in practice}
When the SkillSync idea emerged, I worked through the entire \citet{Reillier2017} toolkit rather than referencing it abstractly. Table~\ref{tab:platform-map} documents the actual entries. The toolkit prompts six questions: who are the producers, who are the consumers, what is the core value unit, which partners support the interaction, what governance rules apply at each step, and how do we measure success. Students act as producers, NGOs as consumers, mentors as partners, and the value unit is a scoped brief matched with a project response. Governance rules include email verification, a scoping wizard, mentor escalation, and dispute templates. Success metrics focus on completion rate, satisfaction, and repeat participation. Writing these answers forced me to operationalise the interaction and now doubles as a draft onboarding checklist.

\begin{table}[H]
  \centering
  \caption{Platform design toolkit worksheet filled for SkillSync based on \citet{Reillier2017}.}
  \label{tab:platform-map}
  \begin{tabular}{p{0.22\linewidth}p{0.33\linewidth}p{0.33\linewidth}}
    \toprule
    Role & Value created and exchanged & Governance guardrail \\
    \midrule
    Students & Verified skill profiles, availability windows, project reflections & Institutional email check, mentor references, code of conduct signoff before browsing briefs \\
    Organisations & Scoped briefs with deliverables, support statements, impact metrics & Wizard enforces clarity on scope, timeline, support, and evaluation criteria before publishing \\
    Mentors and faculty & Feedback comments, escalation paths, reference letters & Moderation rights plus structured retrospectives that log concerns within twenty four hours \\
    Platform team & Matching algorithm, stipend disbursement, analytics dashboards & Data minimisation, opt in analytics reviews, and audit trail reviewed in Assignment~05 \\
    \bottomrule
  \end{tabular}
\end{table}

After the toolkit pass I prototyped the organisation wizard in Figma and prepared hallway testing with two NGOs from previous coursework. The script asked them to fill the template while thinking aloud; the aim was to see whether the prompts prevented scope creep without sounding bureaucratic. Notes from those sessions directly informed Figure~\ref{fig:project-creation}.

\subsection*{Rituals that kept the change grounded}
To avoid another unfounded pivot I documented every experiment. Each fortnight I scored candidate moves against desirability, feasibility, and viability using a simple one to five rubric. If a score dropped below three I paused the idea. For example, the dinner marketplace lost feasibility points because health compliance looked expensive, while SkillSync gained desirability thanks to campus access. I also ran a concierge rehearsal: pairing one NGO brief with two student teams manually to test the workflow. The exercise confirmed that the scoping checklist reduced confusion, so I kept investing in the SkillSync route. These rituals follow \citet{Choudary2016}'s guidance on iterative governance and echo Lecture~6's insistence that pivots should be theory informed rather than impulsive \citep{Lecture06}.

Figure~\ref{fig:project-creation} shows the final wizard draft. Each section corresponds to a governance lever from \citet{Reillier2017}: define the value unit, specify contributions, and clarify rewards. Helper text explains why each field matters so organisations feel coached instead of interrogated.

\begin{figure}[H]
  \centering
  \includegraphics[width=0.85\linewidth]{figures/Organisation-generate-project.png}
  \caption{Organisation project wizard mock up that operationalises the toolkit entries.}
  \label{fig:project-creation}
\end{figure}

\section*{Assignment 04: Monetisation Strategy}
\addcontentsline{toc}{section}{Assignment 04: Monetisation Strategy}

This section converts spreadsheet tests into a monetisation map while still in platform theory instead of wishful thinking.

\subsection*{Revenue options and theoretical grounding}
I listed four revenue streams and matched each principles from \citet{Choudary2016} and \citet{HagiuWright2013}.
\begin{itemize}
  \item \textbf{Completion fee.} A 7\% fee only triggers once both sides confirm delivery. Hagiu and Wright argue that transaction fees should align with realised value to avoid discouraging participation, so the fee stays invisible until success is logged.
  \item \textbf{Enablement subscription.} Larger partners can upgrade to a monthly enablement tier that bundles templated briefs, analytics, and advisory sessions. \citet{Choudary2016} lists producer tools as the second monetisation layer once the core interaction works, so this tier waits until the grow phase.
  \item \textbf{Insight reports.} Skill and impact trends can be sold to universities and municipal innovation teams. \citet{ShapiroVarian1999} note that information goods scale cheaply, yet \citet{Zuboff2019} warns about surveillance, so reports only launch with differential privacy, opt in consent, and the governance checks described in Assignment~05.
  \item \textbf{Grant funded scholarships.} Philanthropic grants and university funds can subsidise stipends for resource constrained NGOs. This follows \citet{ShapiroVarian1999}'s price discrimination guidance and keeps inclusion aligned with Assignment~07's fairness commitments.
\end{itemize}

\subsection*{Sequencing across the lifecycle}
I map timing to the seed, grow, scale arc from \citet{Choudary2016}.
\begin{enumerate}
  \item \textbf{Months 0 to 6.} No fees. The focus is on trust rituals: weekly office hours, onboarding, and a public map. Success means 30 completed projects and satisfaction above four point five.
  \item \textbf{Months 7 to 12.} Introduce the 7\% completion fee for new organisations while \textit{grandfathering} the founding cohort for three months (No fees). Pilot the enablement tier with five agencies and track completion above 85\%, churn below 10\%, and net promoter score above plus thirty. These thresholds come from lecture case studies on balancing monetisation with quality \citep{Lecture05}.
  \item \textbf{Month 12.} Extend fees to all partners, roll out enablement broadly, and launch the insight reports once the data board signs off on privacy safeguards. Monitor revenue per active organisation with engagement to catch risks in extraction \citep{Srnicek2017}.
\end{enumerate}

\subsection*{Experiments and cost discipline}
Each revenue stream is tested through paired tests. The completion fee runs an A/B test comparing when disclosure happens either at brief creation or after matching. The enablement tier uses \textit{Van Westendorp pricing interviews} with at least 20 organisations, and only launches if the acceptable range falls between 400 and 500 DKK. Insight reports begin with a diary study of 15 participants who rate trust on a five-point scale; scores below 3 trigger a redesign. Fixed year-one costs are about 620,000 DKK for product, design, and operations, while variable costs depend on finished sprints. Break-even requires roughly 1,050 projects a year, based on Assignment 09’s capacity model.

Figure~\ref{fig:student-profile} shows the student profile interface that makes these revenue streams plausible. It surfaces badges, skill evidence, and mentor quotes that organisations value, while prompts encourage students to keep data fresh. The mock up acts on \citet{Choudary2016}'s advice to monetise after creating tangible producer surplus.

\begin{figure}[H]
  \centering
  \includegraphics[width=0.8\linewidth]{figures/Student-Profile.png}
  \caption{Student profile mock up highlighting evidence that underpins monetisation.}
  \label{fig:student-profile}
\end{figure}

Ethical rules close the plan. Reports only start once at least five organisations in a sector and five hundred finished projects exist. Consent screens explain why each datapoint is collected, and grants stay separate from transaction fees to avoid bias. These measures turn \citet{Zuboff2019}’s critique into concrete product requirements.
\section*{Assignment 05: Governance and Data Policies}
\addcontentsline{toc}{section}{Assignment 05: Governance and Data Policies}

\subsection*{Onboarding and moderation architecture}
I combined \citet{Choudary2016}'s participation formula with \citet{Reillier2017}'s governance levers to draft the onboarding and moderation plan. Participation requires access, ability, and incentive. Access comes from institution issued logins for students and invitation codes for NGOs. Ability is supported by guided tours, examples, and a first mission checklist. Incentive arrives through portfolio boosts for students and impact dashboards for organisations.

The onboarding journey unfolds in three concrete steps. First, \textbf{a welcome screen} previews sample projects and spells out expectations. Second, \textbf{profile setup} uses defaults for skills, availability, and preferred communication styles. Third, the \textbf{first mission checklist} unlocks only after users read the code of conduct and complete a practice task. Each step ends with a short micro survey so the team collects immediate feedback, reflecting Lecture~5's reminder to treat onboarding as an experiment \citep{Lecture05}.

Moderation uses a layered structure based on \citet{Choudary2016}'s categories of prevention, detection, and enforcement. Automated filters remove clear spam. Community stewards, made up of experienced students and NGO staff, can hide content for a short time and ask for review. A professional response team handles serious cases within twenty four hours, using playbooks created with NGO advisors. Training includes de escalation scripts because previous cases showed that many NGOs work with vulnerable groups, and poor communication can retraumatise participants \citep{Lecture11}. Quarterly tabletop drills check that the process remains effective.

\subsection*{Data policy and transparency}
Data collection follows the \textit{minimalism principle} from \citet{Zuboff2019}. SkillSync only stores information needed for matching, payouts, and quality assurance. This includes profile basics, project actions, satisfaction scores, and optional testimonials. \textit{Advanced analytics} need clear opt in during onboarding. The default data retention period is two years unless users ask for a different duration.

Transparency is implemented through a \textit{data mirror} inside the product. Users can see every \textit{datapoint} linked to their account, download it, or delete it unless a legal hold applies. Release notes share summaries of \textit{algorithm} updates, moderation results, and incident reports every quarter. An internal \textit{ethics council} meets each month to review new experiments. This council uses \citet{Reillier2017}'s \textit{governance canvas} to check if changes shift power unfairly. External advisors from partner NGOs receive the same reports to maintain legitimacy and trust \citep{Srnicek2017}.

Figure~\ref{fig:onboarding-flow} shows the four step carousel that guides onboarding. Each panel uses simple language and includes an indicator explaining why the step is important so users do not feel misled. The design follows the \textit{show not tell} advice from Lecture~5 on metrics and instrumentation \citep{Lecture05}.

\begin{figure}[H]
  \centering
  \begin{minipage}[b]{0.45\textwidth}
    \centering
    \includegraphics[width=\linewidth]{figures/Onboarding-1.png}\\[0.3em]
    \includegraphics[width=\linewidth]{figures/Onboarding-2.png}
  \end{minipage}\hfill
  \begin{minipage}[b]{0.45\textwidth}
    \centering
    \includegraphics[width=\linewidth]{figures/Onboarding-3.png}\\[0.3em]
    \includegraphics[width=\linewidth]{figures/Onboarding-4.png}
  \end{minipage}
  \caption{Guided onboarding carousel with four step checklist.}
  \label{fig:onboarding-flow}
\end{figure}

Governance tooling is shown in Figure~\ref{fig:admin-panel}. The administrator dashboard brings together flagged content, dispute queues, and \textit{fairness metrics} beside response time goals. Moderators can look into case history, use prewritten responses, ask for help from legal counsel, and download evidence for quarterly reports. These functions match the enforcement levers described in \citet{Reillier2017} and make accountability clear.

\begin{figure}[H]
  \centering
  \includegraphics[width=0.85\linewidth]{figures/Organisation-Administratorpanel.png}
  \caption{Governance control room mock up with moderation and fairness metrics.}
  \label{fig:admin-panel}
\end{figure}

Finally, communication remains human. Automated nudges use friendly language and include links to policy explanations. Moderators join workshops with partner NGOs to learn about sensitive situations. Every three months, community sessions let experienced users question the product team before big updates. This follows Lecture~11’s idea of building legitimacy through dialogue \citep{Lecture11}.

\section*{Assignment 06: Competitive Positioning}
\addcontentsline{toc}{section}{Assignment 06: Competitive Positioning}

\subsection*{Landscape and pressure points}
To avoid falling in love with our own idea I mapped the ecosystem using Porter’s five forces \citep{Porter2008}. Direct competitors include Worksome, LinkedIn’s project marketplace, and student-consulting collectives already partnering with NGOs. Substitutes abound: organisations can hire interns, tap volunteer portals, or lean on pro bono consultancies, while students can multi-home on Upwork, hackathons, or extracurricular societies. Supplier power shows up in universities that can revoke access to student communities, so we rely on co-created curricula and data-sharing agreements. Buyer power is meaningful because NGOs operate on tight budgets, so we keep pricing transparent and tie fees to completed value. Threat of new entrants stays high given low technical barriers, making differentiation live in the interaction we choreograph rather than defensive contracts. Multi-homing risk remains the critical pressure point.

\subsection*{Moats we can realistically build}
Platform theory reminds us that sustainable advantage comes from reinforcing loops rather than traditional lock-in \citep{Choudary2016,Reillier2017,Lecture07}. I see three pillars:
\begin{enumerate}
  \item \textbf{Trust-rich matching.} Every project goes through a scoping template and a student--NGO advisory circle, keeping quality high and lowering perceived risk so generic freelance sites struggle to poach matches.
  \item \textbf{Data-powered enablement.} Analytics translate project outcomes into skills passports for students and impact dashboards for NGOs, and the history we capture makes nuanced matchmaking harder to copy \citep{FarrellSaloner1986}.
  \item \textbf{Community partnerships.} We embed with campus career centres and municipal innovation labs that already broker collaborations, creating distribution moats because they trust us with their communities \citep{ShapiroVarian1999}.
\end{enumerate}

\subsection*{Strategic moves}
To operationalise the pillars we commit to three moves: launch a ``project assurance'' programme where we co-pilot the first sprint for new NGOs so reliability and case studies build fast; ship an open API that lets universities sync SkillSync activity into their learning systems, raising switching costs without being anti-competitive; and publish quarterly transparency reports on outcomes, diversity, and data usage to reinforce the ethical stance from Assignment~05 and position us as the legitimacy-first platform \citep{Srnicek2017,Zuboff2019}. The assurance programme doubles as buyer-power mitigation because it lowers perceived risk for NGOs that could otherwise negotiate discounts.

To back the strategy we benchmarked SkillSync against four archetypes: global freelance marketplaces, local student consultancies, university project portals, and specialised volunteer networks. SkillSync wins on curated governance (fast dispute resolution and co-piloted onboarding), ties on project breadth, and deliberately loses on raw scale because we prioritise trust over volume. The comparison exposed a messaging gap, which pushed us to build the chat system in Assignment~07 and reminded us that platform advantage is a systems problem, not a single feature.

A ``red team'' exercise had classmates role-play upstart competitors. One scenario imagined a global tech company cloning us but subsidising NGOs with cloud credits, countered by civic partnerships and proprietary quality data; another pictured universities rebuilding in-house, which we offset with rapid experimentation and analytics insights NGOs rate highly. Documenting the drills maintains strategic discipline by keeping investment focused on relational assets that make copying hard.

\section*{Assignment 07: Inequality and Responsibility}
\addcontentsline{toc}{section}{Assignment 07: Inequality and Responsibility}

I start by mapping who falls through the cracks. Resource-light NGOs lack cash and staff to babysit another dashboard and fear hidden fees or data obligations of the kind \citet{Srnicek2017} critiques. Humanities and design faculties also risk being sidelined because their success metrics differ from the business-school crowd, echoing \citet{Choudary2016}'s reminder that governance must match each segment’s value logic and reinforcing the inequality lens from Lecture~8 \citep{Lecture08}.

To make life easier for NGOs I propose a ``lean onboarding kit'': a ready-made data sheet, templated event briefs, and an access programme where we pair them with students during the first weeks. The VirtuAI case showed how crucial that social onboarding layer is when resources are thin \citep{Gunasilan2024}. Practically it becomes a lightweight flow with clear budget caps and auto-generated reports so organisations skip building measurement tools from scratch.

For faculties the move is to let them shape their own micro-communities. We spin up ``faculty sandboxes'' where humanities define alternative engagement metrics while economists stick with classic growth curves, mirroring \citet{Reillier2017}'s advice on modular governance. We also stay open to analogue experiments that can be documented via simple uploads instead of mandatory livestreams.

Policy-wise I sketch three rules. First, a fairness clause tracks resource spend per organisation and offers fee waivers when volunteer hours pass a threshold, dovetailing with \citet{ShapiroVarian1999}. Second, an inclusion policy grants every faculty a seat on a data-and-ethics council to avoid governance bias, echoing \citet{Zuboff2019} and our Lecture~11 debate \citep{Lecture11}. Third, a recurring impact audit inspired by DineTogether reviews whether features inadvertently favour resource-rich actors each quarter \citep{Rennella2023}.

As an overarching design principle I stick with ``progressive engagement'': the more resources an actor has, the more advanced tools we unlock while the baseline stays simple and free. It operationalises balanced network effects and the pragmatic lessons from our cases so NGOs with minimal budgets and faculties with divergent success criteria can join without feeling overwhelmed, while ambitious partners still see a path to deeper collaboration.

Figure~\ref{fig:chat-system} captures the messaging system behind this fairness work. The polished `Messengersystem.png` interface handles micro-coaching, inclusion triage, and governance updates in plain language. Students can flag ``access support needed'' for quick moderator response, NGOs can request translation help, and templated replies reference our fairness clause so tone stays consistent when moderators rotate.

\begin{figure}[h]
  \centering
  \includegraphics[width=0.85\linewidth]{figures/Messengersystem.png}
  \caption{Messaging workspace (`Messengersystem.png`).}
  \label{fig:chat-system}
\end{figure}

I also introduced a ``mutual aid'' feature where resource-rich partners volunteer surplus capacity (design time, translation, data access) to NGOs with bandwidth gaps. The chat system coordinates offers, logs credits, and routes recognition so inequality does not harden as we scale---a direct response to the gig-work cautionary tales from Lecture~9 \citep{Lecture09}.

\section*{Assignment 08: Metrics and Learning}
\addcontentsline{toc}{section}{Assignment 08: Metrics and Learning}

This section focuses on the metrics that keep SkillSync effective each week rather than on a theoretical appendix.

\subsection*{KPIs that make sense}
\begin{itemize}
    \item \textbf{Matching rate}: Share of suggested matches that land; if it drops we tweak the algorithm or onboarding questions and slice by cohort.
    \item \textbf{Repeat usage rate}: Share of users who return within 30 days; a dip triggers a review of retention features or community rituals.
    \item \textbf{Net Promoter Score}: Signals whether people would recommend us. Drops usually flag fairness issues or bugs, so we pair them with quick interviews.
    \item \textbf{Time-to-first-value}: Minutes to the first meaningful interaction. When it drags, we strip friction or add guided missions.
    \item \textbf{Revenue per active match}: Keeps monetisation tied to behaviour while we monitor variance so a few power users do not prop up the number.
    \item \textbf{Equity of participation}: Share of projects from resource-light partners so we track inclusion goals from Assignment~07.
    \item \textbf{Partner retention}: Share of organisations posting again within 60 days. It keeps the organisation side visible in planning debates.
\end{itemize}

\subsection*{Data infrastructure and feedback loop}
I keep the data stack simple. Events land in a cloud warehouse (BigQuery or Snowflake) because both scale affordably and support granular access controls. We stream via Segment or RudderStack so the app stays decoupled, dbt shapes clean tables for analysis, and dashboards live in Looker Studio or Metabase so anyone can explore without SQL while respecting permissions.

The feedback loop runs on three rhythms, mirroring the instrumentation drills from Lecture~5 on metrics and experimentation \citep{Lecture05}:
\begin{itemize}
    \item \textbf{Weekly reviews}: Product, data, and support meet every Tuesday, walk the KPI dashboard, and check fresh cohorts so onboarding issues surface fast.
    \item \textbf{Monthly cohort analyses}: Segment by acquisition channel and first-match timestamp to see which cohorts stick and pay; the report feeds marketing spend and the roadmap.
    \item \textbf{Quarterly learning readouts}: Summarise experiments, share surprises, and reset hypotheses so the informal student vibe still produces structured knowledge.
\end{itemize}

\subsection*{How metrics guide change}
Imagine matching rate drops from 62\% to 48\% over three weeks. Weekly review shows new users from a partner campaign lag and cohort analysis reveals time-to-first-value above 48 hours. We run an onboarding A/B test, add a preference step, tighten algorithm weights, and ship the winner two sprints later. The next monthly check shows matching back above 60\%, repeat usage up eight points, revenue per active match nudging upward, and equity of participation intact, turning KPIs into a compass instead of decoration.

Figure~\ref{fig:feedback-screen} shows the feedback interface powering these metrics: after each project both sides rate collaboration quality, delivery against scope, and communication cadence while qualitative notes surface for moderation and scores feed the matching algorithm. ``Impact badges'' reinforce good behaviour while the layout keeps the UX light.

\begin{figure}[H]
  \centering
  \includegraphics[width=0.8\linewidth]{Student-Project-Feedback.png}
  \caption{Feedback screen balancing qualitative notes and structured scores.}
  \label{fig:feedback-screen}
\end{figure}

Figure~\ref{fig:feedback-screen} blends notes and scores to power retention.

We built the analytics stack for reproducibility. Dashboards carry ``definition'' tooltips that link to the dbt logic, SQL lives in version control, and a metrics catalogue keeps newcomers oriented. Quarterly KPI snapshots preserve history even as definitions evolve, turning metrics into institutional memory in the spirit of \citet{Choudary2016}.

\section*{Assignment 09: Scaling Strategy}
\addcontentsline{toc}{section}{Assignment 09: Scaling Strategy}

\subsection*{Phase one: proving product market fit}
The first phase happens in one city to keep coordination simple. I aim for sixty active students and twenty finished projects with satisfaction scores above four point five. These targets are based on the launch model in Assignment~02 and \citet{Choudary2016}'s idea of reaching a \textit{minimum viable critical mass} before expanding. Two main partners, ideally a municipal innovation unit and a trusted NGO network, give legitimacy. The team has a product lead, two engineers, a community manager, and part time mentor support. Weekly records every challenge in a shared knowledge base so that all improvements are backed by evidence.

\subsection*{Phase two: regional expansion}
Once the core loops work, the second phase grows across the region. Onboarding flows are turned into standard templates, and the API described in Assignment~06 helps partners connect without custom work. I set up a three tier partner program (community, certified, strategic) with rules for response time, satisfaction, and contribution to \textit{fairness metrics}. The goals increase to two hundred fifty active students, seventy five organisations, and a time to first value under twelve hours. These targets follow \citet{HagiuWright2013}'s focus on balancing growth and quality. A \textit{partner success pod} monitors retention and holds quarterly business reviews, using rituals from enterprise \textit{SaaS} while keeping the tone friendly.

\subsection*{Phase three: national network}
The final phase explores national or Nordic reach once unit economics stabilise. This step needs alliances with national agencies, expanded governance through an advisory board, and white label options for institutions that want their own branding. Success means five hundred projects per year, completion above ninety percent, and net revenue retention above one hundred ten percent. \citet{Srnicek2017} warns that scaling without legitimacy invites backlash, so transparency reports and the inclusion council stay central.

\subsection*{Risk mitigation and decision gates}
Two risks dominate scaling plans: \textit{churn} and \textit{quality drift}. To manage \textit{churn} I track retention in each segment, run exit interviews, and build loyalty through alumni storytelling events. To manage quality I set \textit{service level agreements} for response times, automate match quality checks, and call a community board if satisfaction stays below four point four for two months in a row. If that happens, onboarding of new partners stops until the board approves remediation steps. Figure~\ref{fig:scaling-dashboard} shows the dashboard that monitors these signals.

\begin{figure}[H]
  \centering
  \includegraphics[width=0.75\linewidth]{figures/Student-Dashboard.png}
  \caption{Adoption dashboard mock up tracking activation against partner enablement.}
  \label{fig:scaling-dashboard}
\end{figure}

\subsection*{Scenario modelling}
To test resilience I built a Google Sheets simulation that tracks activation rate, moderation load, partner velocity, and revenue per project. The model estimates staff needs for every one thousand active users and includes a pause rule: if satisfaction stays below four point four for eight weeks, growth spending stops. I also made a downside case where retention drops five points. This reduces annual revenue by about nine million DKK and delays break even by twelve months. Because of this, the playbook includes a cost reduction plan that focuses on core trust features. These exercises follow \citet{Lecture12}'s advice to see scaling as a system design problem instead of a marketing stunt.

\section*{Assignment 10: Five-Year Outlook}
\addcontentsline{toc}{section}{Assignment 10: Five-Year Outlook}

This outlook maintains analytical discipline regarding ambition and risk while pulling together the wrap-up prompts from Lecture~13 \citep{Lecture13}.

\subsection*{Projections}
I mapped a conservative base scenario with three headline numbers, building on the groundwork from earlier assignments:
\begin{itemize}
  \item \textbf{Active users:} 75,000, assuming 55\% annual growth from local cluster launches and retention at 68\% via proactive partner enablement \citep{Choudary2016,Srnicek2017}.
  \item \textbf{Revenue:} 42 million DKK from 7\% transaction fees, data-informed subscriptions, and co-branded partnerships with 58\% gross margin once support is automated \citep{ShapiroVarian1999}.
  \item \textbf{Strategic partnerships:} 18 deals (three national anchors, five sector data hubs, ten municipal or regional innovation units) under governance memoranda \citep{Reillier2017}.
\end{itemize}

The numbers stem from pilot data (conversion around 12\%) and the assumption that by years two and three we automate onboarding so partners can integrate without bespoke development. Sensitivity checks maintain analytical discipline: a five-point drop in retention shrinks the active-user number to 58,000, while delaying automation by a year cuts revenue by roughly 9 million DKK. If viral loops hit harder, usage and revenue might land 30\% higher, but only if community features resonate.

\subsection*{Threats and exit scenarios}
The major threats remain the classics from platform literature: substitution, multi-homing, and regulation. If a global player dumps prices our fees suddenly look expensive \citep{Porter2008}, so we keep a 10\% discount fund, co-marketing agreements, and impact reporting that generic marketplaces rarely match. Transparency failures could let regulators choke data flows \citep{Srnicek2017}, so we publish governance updates, pre-draft responses, and rehearse through tabletop exercises. Multi-homing stays persistent, which is why we protect the integrations and longitudinal insights that make switching costly \citep{FarrellSaloner1986}.

I sketch two realistic exit options if everything collapses: either a controlled acqui-hire where a larger Nordic platform buys the team and IP while we sunset the marketplace, or a pivot into pure data infrastructure that maintains the API layer as SaaS for a smaller client base. Both options demand modular code, clean contracts, quarterly documentation weeks, and escrowed backups \citep{Reillier2017}.

Before we lock the five-year plan, I sketched a ``readiness week'' ritual each June: revisit platform fundamentals and network-effect diagnostics \citep{Lecture01,Lecture02}, rerun monetisation and governance tests \citep{Lecture05,Lecture10}, review inequality and data-ethics metrics with stakeholders \citep{Lecture08,Lecture11}, stage alumni-led red teams \citep{Lecture12}, and close with a go/no-go memo in the spirit of Lecture~13 \citep{Lecture13}. The ritual keeps the outlook a living hypothesis and doubles as onboarding for new hires.

\subsection*{Closing reflection}
This journey is a reminder of how demanding it is to balance growth ambitions with governance. Every design choice hits both sides of the marketplace simultaneously, forcing loop thinking rather than linear funnels \citep{Choudary2016}. Porter’s competitive strategy stays a reality check on differentiation \citep{Porter2008}, and network effects only compound when legitimacy, data quality, and partnerships stay front and centre \citep{Srnicek2017}.

By year five, I expect SkillSync to operate as a civic infrastructure layer where universities plug skill gaps, NGOs find an innovation sandbox, and students see a rite of passage, but only if we keep investing in transparency and monitor Assignment~08's metrics.



\newpage
% Auto bibliography from your references.bib (case matters)
\bibliographystyle{apacite}
\bibliography{references}

\end{document}
