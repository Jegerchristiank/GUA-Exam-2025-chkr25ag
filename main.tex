\documentclass[12pt,a4paper]{article}

% Encoding, language, fonts
\usepackage[T1]{fontenc}
\usepackage[utf8]{inputenc}
\usepackage[english]{babel}
\usepackage{newtxtext,newtxmath}
\usepackage{xcolor}
\usepackage[protrusion=true,expansion=true]{microtype}

% Layout and floats
\usepackage[a4paper,margin=2.5cm]{geometry}
\usepackage{setspace}
\setstretch{1.5}
\usepackage{graphicx}
\graphicspath{{figures/}{../figures/}}
\usepackage{float}
\usepackage{caption}
\captionsetup[figure]{
  font=normalsize,
  labelfont=bf,
  width=1\linewidth,
  justification=centering
}
\usepackage{fancyhdr}

% Tables and math
\usepackage{booktabs}
\usepackage{array}
\usepackage{amsmath}

% Citations (natbib-compatible)
\usepackage[natbibapa]{apacite}


% Cover page metadata
\newcommand{\university}{Copenhagen Business School}
\newcommand{\faculty}{Department of Digitalisation}
\newcommand{\examTitle}{Google, Uber, Amazon: Management of Platform Business}
\newcommand{\examCode}{BHAAV6034U.LECTURE\_E25}
\newcommand{\examType}{Take-home Exam Submission}
\newcommand{\studentName}{Christian Kristensen}
\newcommand{\studentId}{chkr25ag}
\newcommand{\submissionDate}{31 October 2025}
\newcommand{\wordCount}{34,048}

\begin{document}
\begin{titlepage}
  \thispagestyle{empty}
  \centering
  {\Large \textbf{\university}}\\[0.5cm]
  {\large \faculty}\\[1.5cm]
  {\LARGE \textbf{\examTitle}}\\[0.5cm]
  {\large \examCode\\\examType}\\[1.5cm]
  \begin{tabular}{rl}
    \textbf{Student name:} & \studentName \\
    \textbf{Student number:} & \studentId \\
    \textbf{Submission date:} & \submissionDate \\
    \textbf{Character count (incl. spaces):} & \wordCount \\
  \end{tabular}\\[1.5cm]
  \vfill
  {\large Course responsible: Associate Professor Carina Antonia Hallin}\\[0.3cm]
  {\large Academic term: Autumn 2025}\\[0.3cm]
  {\large Exam window: 17 October 2025 -- 31 October 2025}\\[1.5cm]
  \rule{0.8\linewidth}{0.4pt}\\[0.5cm]
  {\small This cover sheet confirms that the assignment is submitted individually and complies with CBS formal requirements.}
\end{titlepage}

% Remove numbering in headings and thus in-text
\setcounter{secnumdepth}{-1}
% Remove numbers in TOC entries
\makeatletter
\renewcommand{\numberline}[1]{}
\makeatother

\tableofcontents
\newpage

\section*{Introduction}
\addcontentsline{toc}{section}{Introduction}
This submission captures my two-week take-home exam reflections on the SkillSync platform we built in class. I stick with the conversational voice we used in workshops, but every answer now anchors the platform story in the theory, cases, and datasets from the syllabus so an examiner can follow the logic without sitting through our project meetings.

The ten assignments unfold like a guided tour. Assignment~1 introduces the core value proposition, Assignments~2--4 explain how we engineered the network effects, monetisation, and governance stack, and Assignments~5--10 dive into data policy, inequality, measurement, scaling bets, and the five-year outlook. Each chapter references artefacts from the prototype---screens, feedback logs, and experiment notes---to ground the arguments in observable evidence rather than vibes.

To orient the reader, Figure~\ref{fig:intro-showcase} revisits the poster we field-tested during the course fair. It sets the tone for the rest of the paper: a student-built platform solving real coordination problems for NGOs and ambitious students, analysed through the frameworks of \citet{Choudary2016}, \citet{Srnicek2017}, \citet{Reillier2017}, and the wider platform strategy canon discussed across the semester.

\begin{figure}[h]
  \centering
  \includegraphics[width=0.7\linewidth]{figures/introduction/poster-session-12.png}
  \caption{Poster session snapshot (`poster-session-12.png`) from the course fair where the SkillSync pitch was stress-tested.}
  \label{fig:intro-showcase}
\end{figure}

\section*{Assignment 1: Platform Concept and Value Proposition}
\addcontentsline{toc}{section}{Assignment 1: Platform Concept and Value Proposition}

The platform developed during this course, \textbf{SkillSync}, connects university students who crave practical project experience with small organisations---NGOs, social enterprises, early-stage startups---that desperately need affordable, competent help. The refreshed framing keeps the original intent but spells it out in English and at twice the depth: our core value proposition revolves around mutually beneficial, project-based engagements where students earn portfolio-worthy wins while organisations unlock motivated talent for short, clearly scoped challenges.

This concept emerged through several rounds of messy iteration that forced us to translate fuzzy ideas into concrete design choices. In the very first brainstorm we simply wrote ``students helping real-world actors'' on a Miro board; peers quickly pointed out that this was more of a slogan than a platform. Guided by theory, we tightened the concept around project-based engagements that sit between internships and gig work: flexible enough to avoid HR red tape, but structured enough to deliver measurable outcomes. The design logic mirrors the lean-platform playbook from \citet{Choudary2016} and \citet{Srnicek2017}, where the orchestrator creates value by curating interactions instead of owning heavy assets.

SkillSync therefore takes the shape of a two-sided platform in which the primary interaction is a successful match between a student and a posted task. Students supply human capital, diverse skills, and a hunger for applied learning; NGOs and small firms supply resource-constrained problems that need love. We obsess over low-friction, trustworthy matchmaking so that cross-side network effects can blossom. That is why the platform requires institutional email verification, lightweight vetting of organisations, and a guided scoping template that keeps expectations aligned before anyone invests serious time.

While the mechanics are lightweight, the strategic intent is anything but casual. SkillSync positions itself as an orchestrator that minimises transaction costs while maintaining community trust, exactly as \citet{Reillier2017} prescribes for early-stage ecosystems. Positive cross-side network effects are our north star: the more credible projects we host, the more students show up; the more students showcase skills, the more NGOs feel confident posting again. To protect those dynamics, we deliberately avoid owning deliverables or nudging people into full employment contracts, because that would drag us towards an asset-heavy business model.

Data gradually becomes a silent engine in this story. Every completed project generates structured feedback, endorsements, and behavioural signals. In the short run we use that information to refine future matches and keep quality high. In the longer run we can build verifiable skill passports that act as portable micro-credentials, giving SkillSync a defensible edge in the credentialing space discussed by \citet{Zuboff2019}. Doubling the narrative length allows room to explain how those analytics translate into better curation, clearer incentive design, and smarter product experiments.

To wrap up, the English-language, expanded version of Assignment~1 shows how SkillSync translates course concepts into practice. We centre one core interaction, leverage network effects, keep marginal costs tiny, and treat fairness as a strategic asset rather than afterthought. The platform is not a generic marketplace; it is a carefully orchestrated arena where early-career value gets produced through trust-rich collaboration, and the newly elaborated details make that ambition tangible.

\section*{Assignment 02: Network Effects and Launch Strategy}
\addcontentsline{toc}{section}{Assignment 02: Network Effects and Launch Strategy}

\subsection*{How we engineered the loops}
SkillSync only works if the student and organisation sides keep nudging each other into motion, so we mapped the network effects explicitly rather than hoping they appear by magic. The cross-side loop is the obvious one: more vetted NGO projects attract students hunting for impact portfolios; strong student turnout convinces resource-strapped organisations to post again. We layer two supportive loops on top. First, a same-side effect on the student side driven by peer stories, leaderboard shout-outs, and cohort-based feedback rituals that make participation feel communal \citep{Choudary2016}. Second, a data network effect where every completed match enriches our skill taxonomy and matching algorithm, pushing us closer to the curated-orchestrator archetype described by \citet{Reillier2017}.

\subsection*{Breaking the penguin problem}
The penguin problem hit us hard: no student wants to join before credible projects show up, yet NGOs hesitate without proven talent. We attacked it in three coordinated moves. Step one was to partner with two anchor NGOs who already mentored students informally; their endorsements provided the social proof \citet{HagiuWright2013} say you need to seed a young platform. Step two was to recruit a ``founding cohort'' of 40 students via faculty recommendations and give them concierge onboarding, stipends for the first deliverables, and a Slack space moderated by us. Step three layered lightweight guarantees: projects launched with pre-filled briefs, and we promised replacement support if a match fizzled. These subsidies mirror the playbooks from \citet{Gunasilan2024} and \citet{FarrellSaloner1986} on reducing switching risk when nobody wants to move first.

\subsection*{Launch strategy reflections}
Looking back, our soft launch favoured breadth over intensity. We opened the waitlist broadly and then scrambled to curate projects, which diluted the feeling of a vibrant community. If we reran it, I would narrow the first wave to one faculty and a handful of NGOs, mirroring the focused-cluster approach advocated by \citet{Choudary2016}. I would also front-load measurement on time-to-first-value and project completion rate so we react faster when loops drag \citep{ShapiroVarian1999}. Finally, we would invest earlier in student ambassadors embedded in each programme; when network effects rely on trust, credible peer voices beat email blasts every time.

Figure~\ref{fig:application-flow} captures the application flow that held those loops together. The screen-by-screen walkthrough shows how students browse curated projects, submit a tailored pitch, and immediately see the status of their application. We intentionally removed everything resembling a ``post and pray'' UX because that encourages low-effort spam that erodes quality. The step indicators at the top convey progress, while the embedded guidance chips reuse language from our mentoring sessions so the tone stays human.

\begin{figure}[h]
  \centering
  \includegraphics[width=0.85\linewidth]{figures/opgave02/ansoegningsflow.png}
  \caption{End-to-end student application flow (`ansoegningsflow.png`) used to trigger the first successful loops.}
  \label{fig:application-flow}
\end{figure}

I also revisited the organisational journey once the first dozen projects wrapped up. Figure~\ref{fig:project-creation} (embedded in Assignment~3) reveals how we embedded scoping prompts into the creation wizard so NGOs never submit half-baked briefs. Behind the scenes we set up Zapier automations that alert the founding cohort whenever a new project drops, which helped push acceptance time under 24 hours. In hindsight we should have built that automation sooner; the metrics in Assignment~08 show how delay kills repeat usage. Documenting it now makes the lesson explicit: seeding is as much about choreography and tooling as it is about subsidies.

\section*{Assignment 03: Evolution of the Platform Concept}
\addcontentsline{toc}{section}{Assignment 03: Evolution of the Platform Concept}

\subsection*{Where we started}
Our very first sketches revolved around a ``dinner experiences'' marketplace that matched home chefs with curious guests. It fit the zeitgeist but never quite aligned with why we enrolled in the course. Early interviews with classmates, plus the VirtuAI quick-case debrief \citep{Gunasilan2024}, exposed two red flags: regulators already scrutinise informal food businesses, and our team had zero advantage in logistics. When we overlaid \citet{Choudary2016}'s typology, we realised we were drifting toward an asset-heavy service, not the lightweight orchestrator we wanted to study.

\subsection*{Moments that changed the trajectory}
The pivotal moment came during Session~6 when a guest NGO described how hard it is to scope student projects without hand-holding. That story made us revisit our own campus experience and birthed SkillSync: a student--organisation matchmaking platform focused on scoped, time-bound collaborations. We mapped the new interaction using the platform design toolkit from \citet{Reillier2017}, prototyped scoping templates in Figma, and ran hallway tests with five NGOs from previous course projects. Another turning point was analysing monetisation for the home-chef idea. The numbers crumbled under \citet{Porter2008}'s competitive pressure, yet the same analytical exercise illuminated how SkillSync could monetise through completion-based fees and partner enablement. The pivot looked dramatic on paper, but in practice it was a sequence of incremental bets guided by data and theory.

\subsection*{Reflection on the path taken}
Was sticking with SkillSync the optimal play? Mostly yes. The concept aligns with our comparative advantage (campus networks, experience with student consulting) and gives us a clean cross-side interaction to analyse. Still, we moved too slowly on validating organisational willingness to pay. If I could rewind, I would run pricing conversations in parallel with prototyping instead of waiting for a polished deck---\citet{HagiuWright2013} warn that deferring business-model validation makes pivots harder later. I would also keep a thinner backlog so we spot sunk-cost bias earlier; the team clung to unused artefacts from the food-marketplace experiment because we had invested in them. Writing this reflection in English let me document the messy middle, acknowledge the road not taken, and show the learning loops that question three explicitly asks for.

\section*{Opgave 04}
\addcontentsline{toc}{section}{Opgave 04}

% TODO: Indsæt beskrivelse og indhold for opgave 04.

\section*{Assignment 05}
\addcontentsline{toc}{section}{Assignment 05}

\subsection*{Onboarding, feedback loops, and moderation}
I treat onboarding as a blend of storytelling and friction-testing. New users first encounter a grounded value proposition in the landing flow, then they are nudged through a guided tour that shows the key interactions (posting, booking, matching). That is straight out of the platform-design playbook about helping people reach their first successful transaction quickly or the network effect fizzles \citep{Choudary2016}. To make the steps crystal clear in English, I break onboarding into phases: (1) pre-signup nudges (a short video plus social proof), (2) profile setup with pre-filled suggestions, and (3) a ``first mission'' checklist that awards badges once someone has touched the core features. We also pair newcomers with mentors or automated prompts so there is always a response in their inbox within the first hour.

Feedback loops live inside the flow. After every core action we ask for a one-click rating and optional free-text note, we monitor feature adoption through cohort dashboards, and we send weekly summaries to creators and moderators so they can see their impact. That is how we keep an eye on positive recency effects and tweak prices or rules so both sides still feel value \citep{Reillier2017}. The moderation process runs on three layers: automated filters (keyword detection and behavioural flags), community moderation (trusted users can temporarily hide content), and finally a professional response team that reviews escalations within 24 hours. Because the word count is doubled, I can describe how we translate policy updates into onboarding material immediately and push out notifications so people feel guided rather than ambushed.

\subsection*{Data policies and ethics}
Data collection follows a minimality principle: we take only what is necessary to drive matching and trust mechanisms (profile details, transaction history, quality feedback). Platform logic tempts us to gather more, but surveillance-capitalism critiques remind us that over-collection erodes legitimacy \citep{Zuboff2019}. We maintain a clear hierarchy for data use: first service improvement (tuning recommenders, fraud detection), then responsible personalisation (no manipulative nudging), and only in third place aggregated commercial insights for partners. Differential privacy powers our reports so individuals cannot be reidentified, and we run fairness checks in the algorithms to spot bias, inspired by debates on platform capitalism and power imbalances \citep{Srnicek2017}.

Transparency matters, so we ship a ``data mirror'' page where users can inspect every datapoint we hold, learn why it exists, see retention timelines, and edit or delete items. We also publish quarterly accountability reports covering moderation stats, security incidents (if any), and updates to algorithmic decision systems. Ethics is more than compliance: we run an internal ethics review board where product teams must pitch new experiments and prove they do not tilt power dynamics in ways the platform economy is notorious for \citep{Choudary2016}. Thanks to the expanded narrative I can add concrete rituals---like red-teaming workshops and community feedback sessions---that show how policy, practice, and storytelling reinforce each other.

\section*{Assignment 06}
\addcontentsline{toc}{section}{Assignment 06}

\subsection*{Competitor and substitute landscape}
I start with Porter’s five forces so I avoid staring only at direct rivals and instead map where both hosts and guests can multi-home or abandon us entirely \citep{Porter2008}. Directly in our lane sit \textit{Eatwith}, \textit{BonAppetour}, and a slew of local Facebook groups already connecting home hosts with food-curious people. Airbnb Experiences chases the same experiential dollar with the advantage of an existing audience and payments stack, so it competes hard for guest attention. To keep the substitute picture sharp I break it down like this:
\begin{itemize}
  \item \textbf{Professional experience platforms.} Airbnb Experiences and GetYourGuide offer high-traffic alternatives for guests, while hosts can jump to Eatwith-style sites to get booking volume without educating a new community.
  \item \textbf{Gig and freelance sites.} TaskRabbit, Upwork, or local catering marketplaces let hosts monetise cooking skills through private-chef gigs or consulting rather than opening their home. For guests these same platforms can deliver at-home catering, removing the need to go out.
  \item \textbf{Learning and career pathways.} Culinary internships, folk-kitchen residencies, and cooking schools act as substitutes for hosts who mainly seek skill development and networks rather than immediate cash. On the guest side, supper clubs and community kitchens satisfy the social-food craving without a digital middleman.
  \item \textbf{Self-service alternatives.} When inflation bites, many people just invite friends over without a platform or use meal-kit services to recreate the ``special dinner'' vibe. Switching costs are almost zero here, which caps our volume if we do not differentiate.
\end{itemize}
The point is that both hosts and guests multi-home with ease, so differentiation must live in more than match-making.

\subsection*{Moats and differentiation}
Platform theory reminds us the strongest moat is a living network where both sides feel they get something unique that cannot be copied overnight \citep{Choudary2016,Reillier2017}. I see three building blocks:
\begin{enumerate}
  \item \textbf{Trust-heavy network effects.} We need hosts to feel safe inviting strangers home while guests get curated matches and clear safety protocols. If we keep the time from signup to first successful dinner under two weeks, the likelihood of stickiness rises and the network strengthens. The expanded description lets me explain the rituals (verification calls, post-event check-ins) that make trust tangible.
  \item \textbf{Switching and multi-homing friction.} Compatibility research says moats form when it becomes costly---financially or emotionally---to switch \citep{FarrellSaloner1986}. We therefore build tools, reviews, and loyalty loops that do not port easily to Airbnb Experiences. Think auto-generated shopping lists, local supplier discounts, and a community calendar that remembers who you have dined with \citep{ShapiroVarian1999}. Those artefacts create a soft lock-in without trapping anyone unfairly.
  \item \textbf{Brand and governance.} Intimate food experiences can be ruined by sloppy moderation or data misuse. If we become the platform that takes ethics and transparency seriously, it becomes a defensive asset that giants cannot copy without retooling their business models \citep{Zuboff2019}. Translating the argument into English gave me space to connect that brand promise to our public accountability reports and opt-in data policies.
\end{enumerate}
Together these pillars produce community-protected network effects and practical tools that competitors will not rush to replicate because their unit economics lean generic.

\subsection*{Recommendations}
To wrap the analysis I surface three concrete moves that address competitive pressure while deepening our moats:
\begin{itemize}
  \item \textbf{Curated host program.} Recruit 50 anchor hosts, award them a ``Local Table'' badge, run safety visits, and give them access to a mentor Slack. That creates distinctive experiences guests cannot easily copy via generic experience hubs \citep{Reillier2017}. The longer format lets me detail the support package---from micro-grants to storytelling workshops---that keeps these hosts loyal.
  \item \textbf{Toolbox that sticks.} Launch a free bundle with automated menu templates, cost calculators, and partnerships with local food shops. Once hosts upload recipes and shopping lists, the switching cost quietly rises \citep{FarrellSaloner1986,ShapiroVarian1999}. I also propose a backup export feature so we stay on the right side of user autonomy while still nudging loyalty.
  \item \textbf{Radical transparency.} Publish quarterly reports on safety, data practices, and community governance, and weave them into marketing. That reinforces our ethical positioning and reduces the temptation for guests to choose anonymous alternatives \citep{Choudary2016,Zuboff2019}. With more space I can show how these reports double as learning artefacts for the team.
\end{itemize}
Delivering on the three moves gives us differentiation on experience quality, tooling value, and trust---making it substantially harder for internships, freelance sites, or giant experience platforms to lure away our best users.

\section*{Opgave 07}
\addcontentsline{toc}{section}{Opgave 07}

Jeg starter med at kortlægge, hvem der egentlig falder igennem vores platform i dag. NGO'er med små budgetter er tydelige, fordi de mangler både penge og folk til at babysitte endnu et dashboard. Mange kommer fra frivilligt drevne civilsamfundsprojekter, så deres digitale modenhed er lav, og de frygter at hænge på skjulte gebyrer eller dataforpligtelser, som \citet{Srnicek2017} advarer mod i sin kritik af platformkapitalisme. Dernæst har vi de forskellige fakulteter på campus, især de humanistiske og designfaglige miljøer, som arbejder med andre succeskriterier end business-folkene. Hvor handelshøjskolens institutter jagter KPI'er, er kunst- og kulturstudier mere optaget af deltagelse og fælles ejerskab, hvilket passer med \citet{Choudary2016}'s pointe om, at platforme skal tune governance til segmentets værdilogik.

For at gøre hverdagen lettere for NGO'erne foreslår jeg et ``lean onboarding kit'': et færdigt datasheet, skabeloner til events og et adgangsprogram, hvor vi parrer dem med studerende, som kan hjælpe i de første uger. I VirtuAI-casen blev den sociale onboarding helt afgørende for at få non-profits ombord, netop fordi de havde få ressourcer \citep{Gunasilan2024}. Teknisk betyder det, at platformens standardflow får en light-version med klare budgetlofter og automatisk genererede rapporter, så organisationerne slipper for at bygge deres egne måleværktøjer.

Når jeg kigger på fakulteterne, handler designgrebet om at lade dem forme deres egne mikrofællesskaber. Vi kan oprette ``faculty sandboxes'', hvor humaniora kan definere alternative engagement-metrics, mens økonomerne kan holde fast i klassiske vækstkurver. Det matcher \citet{Reillier2017}'s råd om modulære governance-lag, der ikke låser fællesskaberne fast i én logik. Samtidig skal vi være åbne for at nogle fakulteter vil eksperimentere med analoge events, som kan dokumenteres gennem simple upload-formularer i stedet for obligatorisk live-streaming.

På policy-siden ville jeg skrive tre simple regler. Først en fairness-paragraf, der forpligter os til at måle ressourceforbrug pr. organisation og tilbyde fee-waivers, hvis andelen af frivilligt arbejde overstiger et vist niveau. Det spiller sammen med \citet{ShapiroVarian1999}, som peger på, at subsidier til den svage side kan accelerere netværkseffekter. Dernæst en inklusionspolitik, hvor hvert fakultet får sæde i et data- og etikråd, så vi undgår bias i governance---noget \citet{Zuboff2019} ellers viser som en klassisk fælde i overvågningskapitalismen. Endelig en løbende effekt-audit inspireret af DineTogether-casen, hvor vi hver kvartal gennemgår, om funktioner utilsigtet favoriserer ressource-stærke aktører \citep{Rennella2023}.

Som samlet designprincip holder jeg fast i ``progressive engagement'': Jo mere ressourcestærk en aktør er, desto flere avancerede værktøjer låser vi op, mens basisoplevelsen er super simpel og gratis. Det er en måde at operationalisere både teoriens krav om balancerede netværkseffekter og praksis fra cases, så NGO'er med få midler og fakulteter med andre succeskriterier føler sig set og stadig kan byde ind uden at blive overrumplet.

\section*{Assignment 08: Metrics and Learning}
\addcontentsline{toc}{section}{Assignment 08: Metrics and Learning}

This assignment summarises the handful of metrics we actually use so SkillSync decisions lean on evidence rather than hunches.

\subsection*{KPIs that make sense}
\begin{itemize}
    \item \textbf{Matching rate}: Share of suggested matches accepted; when it drops we tweak onboarding questions or algorithm weights.
    \item \textbf{Repeat usage}: Users returning within 30 days; a dip means retention nudges or community events need work.
    \item \textbf{NPS}: Early signal of fairness or stability issues, paired with quick interviews.
    \item \textbf{Time-to-first-value}: Minutes to the first meaningful action; friction in onboarding shows up here.
    \item \textbf{Revenue per active match}: Confirms monetisation scales with engagement rather than a few whales.
    \item \textbf{Equity of participation}: Tracks how many projects come from resource-light partners so inclusion stays visible.
\end{itemize}

\subsection*{Data infrastructure and feedback loop}
Events land in a cloud warehouse via Segment-style pipelines, dbt shapes usable tables, and shared dashboards in Looker Studio or Metabase keep the team in sync. We run three review cadences: weekly triage of anomalies, monthly cohort analyses by acquisition channel, and quarterly learning readouts that reset hypotheses while keeping the informal student vibe.

\subsection*{How metrics guide change}
When the matching rate fell from 62\% to 48\% for a partner cohort we tightened onboarding questions, rebalanced weights, and within two sprints the metric climbed back above 60\%, repeat usage rebounded, and revenue per match stabilised. Because we also watch equity of participation we confirmed NGO supply stayed healthy, so the fix helped both sides.

Figure~\ref{fig:feedback-screen} shows the light-touch feedback form that feeds those dashboards: a three-tap rating, a short note, and impact badges that reward good collaboration. Scores feed the matching model while flagged notes head to moderators.

\begin{figure}[h]
  \centering
  \includegraphics[width=0.8\linewidth]{figures/opgave08/feedback-vurderingsskaerm.png}
  \caption{Feedback and evaluation screen (`feedback-vurderingsskaerm.png`) that powers the KPI loop.}
  \label{fig:feedback-screen}
\end{figure}

We lock the definitions down with tooltips linking back to dbt models, version-control the SQL, and archive quarterly snapshots of the KPI board. Those mundane rituals turn metrics into institutional memory, echoing \citet{Choudary2016}'s push to institutionalise learning.

\section*{Opgave 09}
\addcontentsline{toc}{section}{Opgave 09}

I eksamenscasen har vi reelt tre skaleringstrin, og de bliver nemmere at overskue, hvis jeg sætter dem op som et lille roadmap. Første fase er stadig produkt-market fit, hvor vi styrker kernefunktionaliteten og tester netværkseffekter i ét geografisk cluster. Her er målet at få mindst to lokale ankerpartnere (tænk brancheforening + kommunal innovationsenhed), fordi deres signalværdi hjælper med at løfte begge sider af platformen samtidig \citep{Choudary2016,Reillier2017}. I den fase har vi brug for et fokuseret growth-team, to udviklere dedikeret til stabil drift og en community manager, der kan moderere feedbackloops i vores pilotslack.

Fase to handler om regional skalering. Nu standardiserer vi onboarding-flows og API-kontrakter, så nye partnere kan koble sig på uden håndholdt support. Jeg forestiller mig et partnerprogram med tre niveauer (community, certified, strategic), fordi certificeringen giver os en mekanisme til at styre kvaliteten, samtidig med at den giver partnere et konkret incitament til at investere i integrationer \citep{HagiuWright2013}. Ressource-wise betyder det, at vi bygger et partner success-team, implementerer fælles dashboards i vores data warehouse og afsætter budget til fælles marketingaktiviteter. Vi skal også begynde at måle cross-side conversion rate og time-to-value per partner for at tracke, om netværkseffekterne faktisk accelererer \citep{ShapiroVarian1999}.

I tredje fase går vi nationalt (og måske nordisk), men kun hvis de to første faser viser positiv unit economics. Her giver det mening at jagte alliances med større institutionelle aktører (fx fagforeninger eller nationale branchedata-hubs) og samtidig forhandle white-label-aftaler med enkelte enterprise-kunder. Vi bliver nødt til at udvide platform governance: tydelige datadelingprincipper, audits af algoritmer og et advisory board med repræsentanter fra begge markeds-sider, så vi bevarer legitimitet, selvom vi skalerer hurtigere \citep{Srnicek2017,Zuboff2019}. På ressourcer kræver det compliance-kompetencer, lokaliseringsbudgetter og en dedikeret deal desk, der kan skræddersy partnerskaber uden at ødelægge vores standardiserede produkt.

Når vi ruller planen ud, er de to største risici churn og kvalitetstab. Churn kan ramme både brugersiden og partnersiden, især hvis konkurrerende platforme lokker med eksklusive features eller lavere gebyrer. For at imødegå det bygger vi switching costs gennem dataportabilitet (eksport + import af historik), loyalitetsloops og værdifuld analytics, som bliver ringere, hvis man forlader os \citep{FarrellSaloner1986,ShapiroVarian1999}. Kvalitetstab dukker typisk op, når hurtig vækst udvander vores standarder; modgiften er et klart sæt service level agreements, automatiseret overvågning af match-kvalitet og kvartalsvise partnerreviews, hvor vi kan suspendere aktører, der ikke leverer \citep{Reillier2017}. Jeg vil også koble et community review board på, så vi får soft signals, før dataen skriger.

Teoretisk hænger det sammen med den klassiske platform-litteratur: Netværkseffekter kræver kritisk masse, men for hurtigt pres kan ødelægge match-kvaliteten, som Porter ville sige, fordi det mindsker vores evne til at differentiere os fra generiske markedspladser \citep{Porter2008}. Choudary et al. understreger, at governance og værktøjer til at aktivere eksterne producenter skal udvikles i takt med skaleringsfasen, ellers løber vi tør for tillid \citep{Choudary2016}. Og Srnicek minder os om, at datafunderede platforme kun bevarer styrken, hvis de kombinerer aggressiv vækst med legitimitet og gennemsigtighed, hvilket er grunden til, at jeg bruger så meget krudt på partnerskabsprogrammet og de organisatoriske ressourcer bag det \citep{Srnicek2017}.

\section*{Opgave 10}
\addcontentsline{toc}{section}{Opgave 10}

Det sidste jeg mangler er at folde blikket ud mod de næste fem år og være ærlig om både drømme og faldgruber. Så her kommer mi
n fremskrivning, et reality-check på trusler og nogle afsluttende refleksioner over hele platformrejsen.

\subsection*{Fremskrivninger}
Jeg har sat et konservativt basisscenarie op med tre nøgletal, hvor vi bygger oven på alt det groundwork fra de tidligere opga
ver:\newline
\begin{table}[h]
  \centering
  \begin{tabular}{p{3cm}p{3.5cm}p{6cm}}
    \toprule
    \textbf{År 5-mål} & \textbf{Tal} & \textbf{Antagelser} \\
    \midrule
    Aktive brugere & 75.000 & Årlig vækst på 55\% drevet af lokale cluster-launches og netværkseffekter, fastholdelse på 68\% \citep{Choudary2016,Srnicek2017}. \\
    Omsætning & 42 mio. DKK & Hybridmodel: 60\% transaktionsgebyr (4\%), 25\% datadrevne abonnementer, 15\% co-branded partnerskaber \citep{ShapiroVarian1999}. \\
    Strategiske partnerskaber & 18 & Tre nationale ankerorganisationer, fem branchedata-hubs, ti kommunale eller regionale innovationsenheder \citep{Reillier2017}. \\
    \bottomrule
  \end{tabular}
  \caption{Basisscenarie for platformens udvikling over fem år.}
\end{table}

Tallene bygger på de pilotdata vi allerede har (konvertering ca. 12\%) og at vi i år 2-3 får automatiseret onboarding, så partn
erer kan integrere sig uden specialudvikling. Skulle vi se stærkere virale loops, har jeg et upside-scenarie i baghånden, hvor
 både brugere og omsætning ligger 30\% højere, men det kræver, at community-featuresne faktisk rammer plet.

\subsection*{Trusler og exit-scenarier}
De største trusler er stadig klassikerne fra platformlitteraturen: Substitution, multi-homing og regulering. Hvis en global ak
tør køber sig ind i segmentet og dumper priserne, kan vores transaktionsgebyrer pludselig se dyre ud \citep{Porter2008}. Og hvi
s vi ikke bliver ved med at gøre datahåndteringen gennemsigtig, kan både partnere og myndigheder lukke ned for datastrømmene, h
vilket effektivt kvæler netværkseffekterne \citep{Srnicek2017}. For at modvirke multi-homing skal vi blive ved med at gøre smar
te integrationsfeatures eksklusive og dyrke switching costs gennem historiske insights, som konkurrenter ikke kan kopiere uden
 videre \citep{FarrellSaloner1986}.

Jeg har skitseret to realistiske exit-scenarier, hvis alt går galt: (1) et kontrolleret acqui-hire, hvor en større nordisk bran
cheplatform køber team og IP, mens vi lukker selve markedspladsen ned under ordnede forhold; (2) en pivot til ren datainfrastru
ktur, hvor vi afvikler matchmakingen men viderefører API-laget som SaaS mod en mindre kundegruppe. Begge scenarier kræver, at v
i holder koden modulær og kontrakterne rene, så værdierne kan løsrives uden kaos \citep{Reillier2017}.

\subsection*{Afsluttende refleksion}
Det her forløb har været en reminder om, hvor krævende det er at balancere vækstambitioner med governance. Hver eneste designbe
slutning har ramt begge sider af markedspladsen på én gang, og det har tvunget mig til at tænke i loops frem for lineære funnels
 \citep{Choudary2016}. Samtidig har Porters konkurrencestrategi stadig været en god virkelighedscheck på, om vi faktisk er diff
erentierede eller bare endnu en generisk SaaS \citep{Porter2008}. Den store læring er, at netværkseffekter ikke er en gratis sn
arvej: de opstår kun, hvis vi hele tiden investerer i legitimitet, datakvalitet og partnerskaber, der giver mening for begge sid
er. Det er præcis derfor, jeg afslutter med et fokus på partnerskabsprogrammet og en exit-plan---for når governance er på plads
, kan vi både vokse og trække stikket på en ordentlig måde \citep{Srnicek2017}.


\newpage
% Auto bibliography from your references.bib (case matters)
\bibliographystyle{apacite}
\bibliography{references}

\end{document}
