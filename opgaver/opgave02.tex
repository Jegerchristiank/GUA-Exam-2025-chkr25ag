\section*{Assignment 02: Network Effects and Launch Strategy}
\addcontentsline{toc}{section}{Assignment 02: Network Effects and Launch Strategy}

\subsection*{How we engineered the loops}
SkillSync only works if the student and organisation sides keep nudging each other into motion, so we mapped the network effects explicitly rather than hoping they appear by magic. The cross-side loop is the obvious one: more vetted NGO projects attract students hunting for impact portfolios; strong student turnout convinces resource-strapped organisations to post again. We layer two supportive loops on top. First, a same-side effect on the student side driven by peer stories, leaderboard shout-outs, and cohort-based feedback rituals that make participation feel communal \citep{Choudary2016}. Second, a data network effect where every completed match enriches our skill taxonomy and matching algorithm, pushing us closer to the curated-orchestrator archetype described by \citet{Reillier2017}. Translating the strategy into English forces me to name the concrete artefacts---portfolio badges, scoping templates, benchmarking dashboards---that make those loops legible to the team.

\subsection*{Breaking the penguin problem}
The penguin problem hit us hard: no student wants to join before credible projects show up, yet NGOs hesitate without proven talent. We attacked it in three coordinated moves. Step one was to partner with two anchor NGOs who already mentored students informally; their endorsements provided the social proof \citet{HagiuWright2013} say you need to seed a young platform. Step two was to recruit a ``founding cohort'' of 40 students via faculty recommendations and give them concierge onboarding, stipends for the first deliverables, and a Slack space moderated by us. Step three layered lightweight guarantees: projects launched with pre-filled briefs, and we promised replacement support if a match fizzled. These subsidies mirror the playbooks from \citet{Gunasilan2024} and \citet{FarrellSaloner1986} on reducing switching risk when nobody wants to move first. The expanded write-up spells out the operations---office hours, checklists, backup volunteers---that kept the first matches from stalling.

\subsection*{Launch strategy reflections}
Looking back, our soft launch favoured breadth over intensity. We opened the waitlist broadly and then scrambled to curate projects, which diluted the feeling of a vibrant community. If we reran it, I would narrow the first wave to one faculty and a handful of NGOs, mirroring the focused-cluster approach advocated by \citet{Choudary2016}. I would also front-load measurement on time-to-first-value and project completion rate so we react faster when loops drag \citep{ShapiroVarian1999}. Finally, we would invest earlier in student ambassadors embedded in each programme; when network effects rely on trust, credible peer voices beat email blasts every time. Writing this section as a reflective memo keeps the informal, learning-out-loud tone while proving we actually tackled the network-effect mechanics the question asks about.
