\section*{Opgave 02}
\addcontentsline{toc}{section}{Opgave 02}

\subsection*{Idévalg og strategisk forankring}
Vi endte med at vælge en B2B2C-markedsplads for lokale kulturarrangementer, fordi den rammer et ret tydeligt match mellem et fragmenteret udbud (små spillesteder) og et publikum, der leder efter nicheoplevelser. Beslutningen læner sig tungt op ad platformstænkningen fra \citet{Choudary2016,Reillier2017}, hvor værdien ligger i at skabe et flow af interaktioner fremfor at eje indholdet. Samtidig brugte vi læringen fra Quick casen om VirtuAI til at minde os selv om faren ved at forfølge en for bred persona alt for tidligt \citep{Gunasilan2024}. Fordelen ved valget er, at vi kan bygge på eksisterende communities og netværkseffekter uden kæmpe marketingbudget, mens ulempen er, at vi påtager os koordinationen mellem flere sideløbsmarkeder, hvilket historisk har været svært (tænk bare på hvor hårdt eBay kæmpede for nye vertikaler) \citep{HagiuWright2013}. Alternativer som en generisk events-app blev forkastet, fordi de kolliderede med kravet om differentiering i netværksøkonomien \citep{ShapiroVarian1999}.

\subsection*{MVP-afgrænsning og læringsmål}
For MVP'en besluttede vi at fokusere på tre minimale brugerrejser: venues kan oprette events, publikum kan opdage dem via personlig kuratering, og begge kan følge op via en simpel messaging-funktion. Rationalet følger Hagiu \& Wrights fokus på at levere kerneinteraktioner med lav friktion, før man bygger avancerede lag \citep{HagiuWright2013}. Vi satte os eksplicit for at måle “minimum viable participation” frem for klassiske vanity metrics, inspireret af \citet{Reillier2017}. Plusset ved den stramme afgrænsning er, at vi hurtigere kan teste hypoteser om match quality og priselasticitet. Minusset er, at vi udsætter features som event analytics, hvilket kan gøre det sværere at fastholde venues, hvis konkurrenter tilbyder dashboards out of the box. Vi vurderede dog, at analogien til DineTogether-casen viste, at for mange tidlige perks kan mudre værdiforslaget og ramme likviditeten negativt \citep{Rennella2023}.

\subsection*{Fravalgte funktioner og governance-overvejelser}
Vi parkerede både dynamisk ticket pricing og social feed-funktionalitet til en senere fase. Dynamisk pricing kræver fine-grained dataindsamling, som hurtigt bevæger sig over i overvågningslogikker \`a la \citet{Zuboff2019}; det virker overkill, før vi har etableret tillid. Social feed'et blev droppet, fordi erfaringen fra Platform Revolution er, at extraneous features ofte genererer støj og svækker kerneudvekslingen \citep{Choudary2016}. Plusset ved fravalget er et mere fokuseret produkt og færre governance-brud, særligt når vi endnu ikke har klare retningslinjer for moderering. Omvendt går vi glip af potentielle krydseffekter, som Shapiro \& Varian peger på kan styrke switching costs \citep{ShapiroVarian1999}. Vi tager dog den byrde senere, når vi kan definere datapolitikker og incitamenter uden at skyde os selv i foden.
