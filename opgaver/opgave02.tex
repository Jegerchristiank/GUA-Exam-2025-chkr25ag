\section*{Assignment 02}
\addcontentsline{toc}{section}{Assignment 02}

\subsection*{Idea selection and strategic grounding}
We ultimately chose to build a B2B2C marketplace for local cultural events because it bridges a glaring disconnect between fragmented venue supply and audiences hunting for niche experiences. Translating the rationale into English (and expanding it) highlights how the decision sits squarely within the platform strategy frameworks from \citet{Choudary2016} and \citet{Reillier2017}: our job is to orchestrate a steady flow of interactions, not to hoard content. We leaned heavily on the VirtuAI quick-case debrief from \citet{Gunasilan2024} as a warning against chasing an overly broad persona too early in the journey. The upside of this idea is that we can piggyback on existing communities and cross-side network effects without torching a marketing budget; the downside is that we must coordinate several parallel markets, a famously tricky feat if we remember how hard eBay struggled to expand into new verticals \citep{HagiuWright2013}.

Doubling the narrative length lets me unpack the alternatives we discarded. A generic event-discovery app was tempting because it looked larger on paper, but it clashed with the need for differentiation in network economies \citep{ShapiroVarian1999}. Likewise, pivoting into ticketing infrastructure would have distracted us with compliance work and eroded the lightweight matchmaking thesis. By situating the idea inside real user stories---artists needing pop-up spaces, fans craving genre-specific nights, municipal culture offices searching for partners---we build conviction that the marketplace format can deliver repeatable value without pretending to solve the entire live-entertainment stack on day one.

\subsection*{MVP scope and learning goals}
For the MVP we focus on three minimal user journeys: venues can post events, audiences can discover them via personal curation, and both sides can follow up through a simple messaging flow. The logic follows \citet{HagiuWright2013}'s insistence on delivering the core interaction with minimal friction before layering fancy extras. We explicitly measure ``minimum viable participation'' instead of vanity metrics, taking a cue from \citet{Reillier2017}. The benefit of the tight scope is that we can rapidly test hypotheses about match quality, pricing, and cultural fit. The drawback is that we postpone analytics dashboards, which makes retention harder if competitors dangle instant insights to venues.

With extra space I can spell out the safeguards we add to compensate. First, we establish weekly check-ins with ten pilot venues to gather qualitative feedback and catch edge cases our metrics would miss. Second, we build a lightweight waitlist for superfans who want early tickets, using it to validate willingness-to-pay experiments without bloating the main product. Third, we document each experiment as a narrative memo so the whole team can revisit why a decision worked or failed---a scrappy governance practice that keeps us aligned with the informal, student-like voice this deliverable aims for.

\subsection*{Deferred features and governance considerations}
We parked dynamic ticket pricing and social-feed functionality for later phases. Dynamic pricing demands granular data collection that easily slides into surveillance logics \`a la \citet{Zuboff2019}; it feels like overkill before we build trust. The social feed was cut because \citet{Choudary2016}'s take from \emph{Platform Revolution} reminds us that extraneous features create noise and weaken the core exchange. The plus side of deferring is a sharper product with fewer governance breaches, especially while our moderation playbook is still forming. The minus side is losing potential cross-side synergies that could raise switching costs, something \citet{ShapiroVarian1999} argue can differentiate information goods once scale kicks in.

To make the trade-offs explicit, we line up conditions for revisiting each feature. Dynamic pricing returns to the roadmap only if venues request it, our data policy covers consent and retention, and we can model its effect on fairness for independent artists. The social feed stays on ice until the community itself starts producing content worth curating and we have clear guidelines for spotlighting without amplifying bias. By narrating these governance guardrails in English and in detail, the assignment now shows that restraint can be a strategic move, not just a resource constraint.
