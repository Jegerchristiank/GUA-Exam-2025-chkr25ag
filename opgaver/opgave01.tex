\section*{Assignment 01: Platform Concept and Value Proposition}
\addcontentsline{toc}{section}{Assignment 01: Platform Concept and Value Proposition}

SkillSync links students who want meaningful work with NGOs that lack paid capacity. The offer is simple: scoped projects that swap student portfolio gains for real organisational progress. After a few false starts we anchored the idea in \citet{Choudary2016} and \citet{Srnicek2017}, framing SkillSync as an orchestrator that keeps coordination light but outcomes tangible.

Cross-side trust is the product. Students verify with institutional emails, NGOs complete a plain-language scoping wizard, and both sides see the same expectation checklist before accepting a match. Every completed project feeds ratings and notes into the matching algorithm so quality improves without heavy staffing.

Figure~\ref{fig:student-view} shows the student dashboard emphasising clear scope, mentor nudges, and time commitments. NGOs view a mirrored flow that guides them through deliverables and support upfront, reducing the misfires that scared partners early on. With those pieces in place the value proposition moves beyond a pitch line and becomes a repeatable, low-friction collaboration system.

\begin{figure}[h]
  \centering
  \includegraphics[width=0.85\linewidth]{figures/opgave01/projektvisning-student.png}
  \caption{Student project view (`projektvisning-student.png`) that operationalises the SkillSync value proposition.}
  \label{fig:student-view}
\end{figure}
