\section*{Opgave 1: Platform Concept and Value Proposition}
\addcontentsline{toc}{section}{Opgave 1: Platform Concept and Value Proposition}

The platform developed during this course, \textbf{SkillSync}, aims to connect university students seeking practical project experience with small organizations such as NGOs or startups in need of affordable, competent help. The platform's core value proposition centers on mutually beneficial, project-based engagements: students receive hands-on experience to strengthen their CVs and skillsets, while organizations obtain qualified, motivated talent for short-term challenges.

This concept emerged through iterative refinement, aligning with the pedagogical objective of connecting abstract theory to concrete design. Initially, the idea was loosely structured around ``students helping real-world actors,'' but through critical analysis and peer feedback, the focus sharpened toward project-based engagements that are neither internships nor gig work---but something more flexible, situated between education and employment.

SkillSync is structured as a two-sided platform, where the primary interaction is the successful matching of a student to a task posted by an NGO or small enterprise. In the language of \citet{Choudary2016}, the platform facilitates a core transaction that is value-generating for both sides, while avoiding asset-heavy or employment-like obligations. The platform is lean, modular, and driven by user participation, consistent with the typology of lean platforms proposed by \citet{Srnicek2017}.

From a strategic perspective, the platform's user groups are distinct yet interdependent. Students form the supply side; they bring human capital, diversity of skills, and a desire for applied learning. NGOs form the demand side; they bring underfunded but high-impact project needs. The key to SkillSync’s viability lies in enabling trustworthy, low-friction matches between these sides. This was foregrounded in the design logic: students verify themselves using institutional emails, and organizations undergo lightweight vetting to ensure credibility.

This dual-sided structure emphasizes SkillSync’s mediating role. Rather than directly managing labor or content, the platform facilitates the discovery, initiation, and review of project-based work. In platform strategy terms, SkillSync acts as an orchestrator---not a producer nor a consumer---and relies on positive cross-side network effects to scale. The theoretical grounding from \citet{Reillier2017} reinforces this: platforms that minimize friction while maintaining trust are best positioned to gain early adoption and sustained participation.

Finally, the platform creates long-term value through data. While not monetized in early stages, the system records project feedback, skill endorsements, and behavioral data. This not only helps refine future matches but lays the foundation for features such as verifiable skill portfolios---a potential long-term differentiator consistent with the networked learning and credentialing paradigms discussed in \citet{Zuboff2019}.

In summary, the conceptualization of SkillSync integrates core course insights into platform design: it focuses on one primary interaction, leverages network effects, minimizes marginal costs, and preserves fairness. It is not just a matchmaking tool, but a structured space where early-career value is produced through carefully curated collaboration.
