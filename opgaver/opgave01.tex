\section*{Assignment 1: Platform Concept and Value Proposition}
\addcontentsline{toc}{section}{Assignment 1: Platform Concept and Value Proposition}

The platform developed during this course, \textbf{SkillSync}, connects university students who crave practical project experience with small organisations---NGOs, social enterprises, early-stage startups---that desperately need affordable, competent help. The refreshed framing keeps the original intent but spells it out in English and at twice the depth: our core value proposition revolves around mutually beneficial, project-based engagements where students earn portfolio-worthy wins while organisations unlock motivated talent for short, clearly scoped challenges.

This concept emerged through several rounds of messy iteration that forced us to translate fuzzy ideas into concrete design choices. In the very first brainstorm we simply wrote ``students helping real-world actors'' on a Miro board; peers quickly pointed out that this was more of a slogan than a platform. Guided by theory, we tightened the concept around project-based engagements that sit between internships and gig work: flexible enough to avoid HR red tape, but structured enough to deliver measurable outcomes. The design logic mirrors the lean-platform playbook from \citet{Choudary2016} and \citet{Srnicek2017}, where the orchestrator creates value by curating interactions instead of owning heavy assets.

SkillSync therefore takes the shape of a two-sided platform in which the primary interaction is a successful match between a student and a posted task. Students supply human capital, diverse skills, and a hunger for applied learning; NGOs and small firms supply resource-constrained problems that need love. We obsess over low-friction, trustworthy matchmaking so that cross-side network effects can blossom. That is why the platform requires institutional email verification, lightweight vetting of organisations, and a guided scoping template that keeps expectations aligned before anyone invests serious time.

While the mechanics are lightweight, the strategic intent is anything but casual. SkillSync positions itself as an orchestrator that minimises transaction costs while maintaining community trust, exactly as \citet{Reillier2017} prescribes for early-stage ecosystems. Positive cross-side network effects are our north star: the more credible projects we host, the more students show up; the more students showcase skills, the more NGOs feel confident posting again. To protect those dynamics, we deliberately avoid owning deliverables or nudging people into full employment contracts, because that would drag us towards an asset-heavy business model.

Data gradually becomes a silent engine in this story. Every completed project generates structured feedback, endorsements, and behavioural signals. In the short run we use that information to refine future matches and keep quality high. In the longer run we can build verifiable skill passports that act as portable micro-credentials, giving SkillSync a defensible edge in the credentialing space discussed by \citet{Zuboff2019}. Doubling the narrative length allows room to explain how those analytics translate into better curation, clearer incentive design, and smarter product experiments.

To wrap up, the English-language, expanded version of Assignment~1 shows how SkillSync translates course concepts into practice. We centre one core interaction, leverage network effects, keep marginal costs tiny, and treat fairness as a strategic asset rather than afterthought. The platform is not a generic marketplace; it is a carefully orchestrated arena where early-career value gets produced through trust-rich collaboration, and the newly elaborated details make that ambition tangible.
