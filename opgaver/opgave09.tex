\section*{Opgave 09}
\addcontentsline{toc}{section}{Opgave 09}

I eksamenscasen har vi reelt tre skaleringstrin, og de bliver nemmere at overskue, hvis jeg sætter dem op som et lille roadmap. Første fase er stadig produkt-market fit, hvor vi styrker kernefunktionaliteten og tester netværkseffekter i ét geografisk cluster. Her er målet at få mindst to lokale ankerpartnere (tænk brancheforening + kommunal innovationsenhed), fordi deres signalværdi hjælper med at løfte begge sider af platformen samtidig \citep{Choudary2016,Reillier2017}. I den fase har vi brug for et fokuseret growth-team, to udviklere dedikeret til stabil drift og en community manager, der kan moderere feedbackloops i vores pilotslack.

Fase to handler om regional skalering. Nu standardiserer vi onboarding-flows og API-kontrakter, så nye partnere kan koble sig på uden håndholdt support. Jeg forestiller mig et partnerprogram med tre niveauer (community, certified, strategic), fordi certificeringen giver os en mekanisme til at styre kvaliteten, samtidig med at den giver partnere et konkret incitament til at investere i integrationer \citep{HagiuWright2013}. Ressource-wise betyder det, at vi bygger et partner success-team, implementerer fælles dashboards i vores data warehouse og afsætter budget til fælles marketingaktiviteter. Vi skal også begynde at måle cross-side conversion rate og time-to-value per partner for at tracke, om netværkseffekterne faktisk accelererer \citep{ShapiroVarian1999}.

I tredje fase går vi nationalt (og måske nordisk), men kun hvis de to første faser viser positiv unit economics. Her giver det mening at jagte alliances med større institutionelle aktører (fx fagforeninger eller nationale branchedata-hubs) og samtidig forhandle white-label-aftaler med enkelte enterprise-kunder. Vi bliver nødt til at udvide platform governance: tydelige datadelingprincipper, audits af algoritmer og et advisory board med repræsentanter fra begge markeds-sider, så vi bevarer legitimitet, selvom vi skalerer hurtigere \citep{Srnicek2017,Zuboff2019}. På ressourcer kræver det compliance-kompetencer, lokaliseringsbudgetter og en dedikeret deal desk, der kan skræddersy partnerskaber uden at ødelægge vores standardiserede produkt.

Når vi ruller planen ud, er de to største risici churn og kvalitetstab. Churn kan ramme både brugersiden og partnersiden, især hvis konkurrerende platforme lokker med eksklusive features eller lavere gebyrer. For at imødegå det bygger vi switching costs gennem dataportabilitet (eksport + import af historik), loyalitetsloops og værdifuld analytics, som bliver ringere, hvis man forlader os \citep{FarrellSaloner1986,ShapiroVarian1999}. Kvalitetstab dukker typisk op, når hurtig vækst udvander vores standarder; modgiften er et klart sæt service level agreements, automatiseret overvågning af match-kvalitet og kvartalsvise partnerreviews, hvor vi kan suspendere aktører, der ikke leverer \citep{Reillier2017}. Jeg vil også koble et community review board på, så vi får soft signals, før dataen skriger.

Teoretisk hænger det sammen med den klassiske platform-litteratur: Netværkseffekter kræver kritisk masse, men for hurtigt pres kan ødelægge match-kvaliteten, som Porter ville sige, fordi det mindsker vores evne til at differentiere os fra generiske markedspladser \citep{Porter2008}. Choudary et al. understreger, at governance og værktøjer til at aktivere eksterne producenter skal udvikles i takt med skaleringsfasen, ellers løber vi tør for tillid \citep{Choudary2016}. Og Srnicek minder os om, at datafunderede platforme kun bevarer styrken, hvis de kombinerer aggressiv vækst med legitimitet og gennemsigtighed, hvilket er grunden til, at jeg bruger så meget krudt på partnerskabsprogrammet og de organisatoriske ressourcer bag det \citep{Srnicek2017}.
