\section*{Assignment 06}
\addcontentsline{toc}{section}{Assignment 06}

\subsection*{Competitor and substitute landscape}
I start with Porter’s five forces so I avoid staring only at direct rivals and instead map where both hosts and guests can multi-home or abandon us entirely \citep{Porter2008}. Directly in our lane sit \textit{Eatwith}, \textit{BonAppetour}, and a slew of local Facebook groups already connecting home hosts with food-curious people. Airbnb Experiences chases the same experiential dollar with the advantage of an existing audience and payments stack, so it competes hard for guest attention. To keep the substitute picture sharp I break it down like this:
\begin{itemize}
  \item \textbf{Professional experience platforms.} Airbnb Experiences and GetYourGuide offer high-traffic alternatives for guests, while hosts can jump to Eatwith-style sites to get booking volume without educating a new community.
  \item \textbf{Gig and freelance sites.} TaskRabbit, Upwork, or local catering marketplaces let hosts monetise cooking skills through private-chef gigs or consulting rather than opening their home. For guests these same platforms can deliver at-home catering, removing the need to go out.
  \item \textbf{Learning and career pathways.} Culinary internships, folk-kitchen residencies, and cooking schools act as substitutes for hosts who mainly seek skill development and networks rather than immediate cash. On the guest side, supper clubs and community kitchens satisfy the social-food craving without a digital middleman.
  \item \textbf{Self-service alternatives.} When inflation bites, many people just invite friends over without a platform or use meal-kit services to recreate the ``special dinner'' vibe. Switching costs are almost zero here, which caps our volume if we do not differentiate.
\end{itemize}
The point is that both hosts and guests multi-home with ease, so differentiation must live in more than match-making.

\subsection*{Moats and differentiation}
Platform theory reminds us the strongest moat is a living network where both sides feel they get something unique that cannot be copied overnight \citep{Choudary2016,Reillier2017}. I see three building blocks:
\begin{enumerate}
  \item \textbf{Trust-heavy network effects.} We need hosts to feel safe inviting strangers home while guests get curated matches and clear safety protocols. If we keep the time from signup to first successful dinner under two weeks, the likelihood of stickiness rises and the network strengthens. The expanded description lets me explain the rituals (verification calls, post-event check-ins) that make trust tangible.
  \item \textbf{Switching and multi-homing friction.} Compatibility research says moats form when it becomes costly---financially or emotionally---to switch \citep{FarrellSaloner1986}. We therefore build tools, reviews, and loyalty loops that do not port easily to Airbnb Experiences. Think auto-generated shopping lists, local supplier discounts, and a community calendar that remembers who you have dined with \citep{ShapiroVarian1999}. Those artefacts create a soft lock-in without trapping anyone unfairly.
  \item \textbf{Brand and governance.} Intimate food experiences can be ruined by sloppy moderation or data misuse. If we become the platform that takes ethics and transparency seriously, it becomes a defensive asset that giants cannot copy without retooling their business models \citep{Zuboff2019}. Translating the argument into English gave me space to connect that brand promise to our public accountability reports and opt-in data policies.
\end{enumerate}
Together these pillars produce community-protected network effects and practical tools that competitors will not rush to replicate because their unit economics lean generic.

\subsection*{Recommendations}
To wrap the analysis I surface three concrete moves that address competitive pressure while deepening our moats:
\begin{itemize}
  \item \textbf{Curated host program.} Recruit 50 anchor hosts, award them a ``Local Table'' badge, run safety visits, and give them access to a mentor Slack. That creates distinctive experiences guests cannot easily copy via generic experience hubs \citep{Reillier2017}. The longer format lets me detail the support package---from micro-grants to storytelling workshops---that keeps these hosts loyal.
  \item \textbf{Toolbox that sticks.} Launch a free bundle with automated menu templates, cost calculators, and partnerships with local food shops. Once hosts upload recipes and shopping lists, the switching cost quietly rises \citep{FarrellSaloner1986,ShapiroVarian1999}. I also propose a backup export feature so we stay on the right side of user autonomy while still nudging loyalty.
  \item \textbf{Radical transparency.} Publish quarterly reports on safety, data practices, and community governance, and weave them into marketing. That reinforces our ethical positioning and reduces the temptation for guests to choose anonymous alternatives \citep{Choudary2016,Zuboff2019}. With more space I can show how these reports double as learning artefacts for the team.
\end{itemize}
Delivering on the three moves gives us differentiation on experience quality, tooling value, and trust---making it substantially harder for internships, freelance sites, or giant experience platforms to lure away our best users.
