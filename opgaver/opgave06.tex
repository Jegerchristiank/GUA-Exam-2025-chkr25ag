\section*{Assignment 06: Competitive Positioning}
\addcontentsline{toc}{section}{Assignment 06: Competitive Positioning}

\subsection*{Landscape and pressure points}
To keep ourselves from falling in love with our own idea, I mapped the ecosystem using Porter’s five forces \citep{Porter2008}. Direct competitors include Worksome, LinkedIn’s project marketplace, and smaller student-consulting collectives already partnering with NGOs. Substitute options abound: organisations can hire interns, tap volunteer portals like VolunteerMatch, or partner with consultancies for pro bono sprints. Students, meanwhile, can multi-home on Upwork, join hackathons, or focus on extracurricular societies that deliver similar portfolio value. Multi-homing risk is sky high, so differentiation must live in the interaction we choreograph, not in defensive contracts.

\subsection*{Moats we can realistically build}
Platform theory reminds us that sustainable advantage comes from reinforcing loops rather than traditional lock-in \citep{Choudary2016,Reillier2017}. I see three pillars:
\begin{enumerate}
  \item \textbf{Trust-rich matching.} Every project goes through a scoping template and is reviewed by a student--NGO advisory circle. That keeps quality high and lowers perceived risk for both sides, making it harder for generic freelance sites to poach our best matches.
  \item \textbf{Data-powered enablement.} We invest in analytics that translate project outcomes into skills passports for students and impact dashboards for NGOs. The more history we capture, the harder it becomes for rivals to replicate the nuanced matchmaking without years of data, echoing \citet{FarrellSaloner1986}'s take on compatibility advantages.
  \item \textbf{Community partnerships.} We embed with campus career centres and municipal innovation labs that already broker collaborations. Those partnerships act as distribution moats because they trust us with their communities, a softer barrier \citet{ShapiroVarian1999} say often trumps hard technology advantages.
\end{enumerate}

\subsection*{Strategic moves}
To operationalise the pillars we commit to three actions. First, launch a ``project assurance'' programme where we co-pilot the first sprint for new NGOs; it signals reliability and creates case studies we can reuse. Second, build an open API that lets universities sync SkillSync activity into their learning management systems, raising switching costs without being anti-competitive. Third, publish quarterly transparency reports on project outcomes, diversity metrics, and data usage. That reinforces the ethical stance we discussed in Assignment~05 and positions us as the platform that takes legitimacy seriously \citep{Srnicek2017,Zuboff2019}. Writing it out in this informal tone helps me remember the playbook while keeping the analysis grounded in reality.
