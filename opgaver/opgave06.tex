\section*{Opgave 06}
\addcontentsline{toc}{section}{Opgave 06}

\subsection*{Konkurrent- og substitutkort}
Jeg starter med at ramme banen op via Porters fem kræfter, så jeg ikke kun ser på direkte rivaler, men også de steder hvor både værter og gæster kan multi-home eller helt droppe os.\citep{Porter2008} På den direkte akse ligger \textit{Eatwith}, \textit{BonAppetour} og en håndfuld lokale Facebook-grupper, som allerede forbinder hjemmeværter med madnysgerrige. Airbnb Experiences jager den samme oplevelseskrone, men bringer et massivt eksisterende publikum og betalingsinfrastruktur, så de presser os i kampen om gæsternes opmærksomhed. For at holde styr på de reelle substitutter deler jeg dem op sådan her:
\begin{itemize}
  \item \textbf{Professionelle oplevelsesplatforme.} Airbnb Experiences og GetYourGuide er high-traffic alternativer for gæster, mens værter kan hoppe til Eatwith-lignende sites for at få bookingvolumen uden at skulle uddanne et nyt community.
  \item \textbf{Gig- og freelancesites.} TaskRabbit, Upwork eller lokale cateringmarkedspladser giver værter mulighed for at monetisere madskills gennem cateringjobs, privat kokkehjælp eller konsulentopgaver i stedet for at åbne hjemmet. For gæsterne kan de samme platforme levere hjemmecatering, så de slipper for at tage ud.
  \item \textbf{Læring og karriereveje.} Kulinariske internships, folkekøkken- og højskoleophold er substitutter for værter, der i bund og grund søger kompetenceudvikling og netværk frem for direkte monetisering. På gæstesiden kan madklubber, supper clubs og community kitchens opfylde det sociale madbehov uden en digital mellemmand.
  \item \textbf{Self-service alternativer.} Når inflation bider, vælger mange at invitere vennerne hjem uden platform eller bruge måltidskasser til at genskabe "oplevelsen". Den kategori presser vores volumen, fordi switching-kostnaden er tæt på nul.
\end{itemize}
Pointen er, at både værter og gæster har let ved at multi-home, så vores differentiering skal ligge i mere end blot match-making.

\subsection*{Moats og differentiering}
Platformteori minder os om, at den stærkeste moat ofte er et levende netværk, hvor begge sider føler, de får noget unikt, som ikke lige kan kopieres.\citep{Choudary2016,Reillier2017} Jeg ser tre byggesten:
\begin{enumerate}
  \item \textbf{Tillidsfulde netværkseffekter.} Vi skal gøre værter trygge ved at åbne hjemmet, mens gæster får kuraterede matches og tydelige sikkerhedsprotokoller. Hvis vi kan holde tiden fra signup til første vellykkede middag under to uger, stiger sandsynligheden for, at de bliver hængende og booster netværkseffekten.
  \item \textbf{Switching- og multi-homing friktion.} Ifølge kompatibilitetslitteraturen er det stærkt, når vi gør det dyrt (økonomisk eller emotionelt) at skifte platform.\citep{FarrellSaloner1986} Her handler det om at bygge værktøjer, reviews og loyalitet, der ikke følger med over til Airbnb Experiences. Tænk auto-genererede indkøbslister, lokal leverandør-rabatter og et community-kalender, der viser hvem man har spist med.\citep{ShapiroVarian1999}
  \item \textbf{Brand og governance.} En intim madoplevelse kan ødelægges af dårlig moderation eller datamisbrug. Hvis vi kan være platformen, der eksplicit tager stilling til etik og transparens, er det et defensivt asset, som er sværere for giganterne at kopiere uden at ændre deres forretningsmodel.\citep{Zuboff2019}
\end{enumerate}
Samlet set giver det os en kombination af community-beskyttede netværkseffekter og praktiske værktøjer, som konkurrenterne ikke gider bygge med det samme, fordi deres unit economics er mere generiske.

\subsection*{Anbefalinger}
For at lukke analysen laver jeg tre konkrete træk, der både adresserer konkurrencetrykket og bygger moats:
\begin{itemize}
  \item \textbf{Kurateret værtsprogram.} Rekruttér 50 nøgleværter og giv dem et "Local Table" badge, sikkerhedsbesøg og adgang til en mentor-slack. Det skaber eksklusive oplevelser, som gør, at gæster ikke bare kan kopiere det via generiske oplevelseshubs.\citep{Reillier2017}
  \item \textbf{Værktøjskasse der binder.} Lancér et gratis bundle med automatiske menu-templates, cost-calculator og partnerskaber med lokale fødevarebutikker. Når værter først har lagt deres opskrifter og indkøbslister ind, er switching-kostnaden pludselig højere.\citep{FarrellSaloner1986,ShapiroVarian1999}
  \item \textbf{Radikal transparens.} Kommunikér kvartalsvise rapporter om sikkerhed, data og community governance, og gør dem til en del af marketing. Det spiller direkte ind i vores etiske positionering og dæmper risikoen for, at gæster vælger mere anonyme alternativer.\citep{Choudary2016,Zuboff2019}
\end{itemize}
Hvis vi leverer på de tre trin, står vi med en tydelig differentiering på både oplevelse, værktøjsværdi og tillid, hvilket gør det markant sværere for både internships, freelancesites og oplevelsesplatforme at lokke vores bedste brugere væk.
