\section*{Assignment 04: Monetisation Strategy}
\addcontentsline{toc}{section}{Assignment 04: Monetisation Strategy}

In this question I translate our messy spreadsheets into a coherent plan for how SkillSync earns money without suffocating the fragile network effects we are still nurturing.

\subsection*{Revenue streams on the table}
\begin{itemize}
  \item \textbf{Completion fee on each project.} Organisations pay 7\% of the agreed stipend once a project is delivered and rated successful by both sides. It keeps monetisation anchored to the value we already orchestrate, just as \citet{HagiuWright2013} recommend. The fee stays invisible to students so they do not feel priced out.
  \item \textbf{Enablement subscription for organisations.} Larger NGOs and public agencies can upgrade to a "Partner" tier (499 DKK/month) unlocking templated briefs, analytics, and a dedicated coach. This mirrors \citet{Choudary2016}'s argument that platforms should invest in producer tools that raise quality.
  \item \textbf{Talent insights add-on.} Once we accumulate enough data we can sell aggregated skill trend reports to universities and municipal innovation units. We guard against the surveillance trap \citet{Zuboff2019} warns about by applying differential privacy and obtaining explicit consent before monetising any aggregated insights.
  \item \textbf{Grant-backed scholarships.} We plan to pursue social-impact grants that subsidise student stipends for NGOs who cannot afford the completion fee. It is not a profit centre, but it diversifies funding and keeps the marketplace inclusive, aligning with \citet{ShapiroVarian1999}'s note that subsidising one side can accelerate network effects.
\end{itemize}

\subsection*{Timing across the platform lifecycle}
I anchor the roadmap in the classic seed $\rightarrow$ grow $\rightarrow$ harvest logic \citep{Choudary2016}.
\begin{enumerate}
  \item \textbf{Seeding (0--6 months).} No fees, just memoranda of understanding that outline future pricing. We focus on hitting 30 completed projects and building trust rituals (weekly office hours, templated retrospectives).
  \item \textbf{Early growth (months 7--12).} Introduce the 7\% completion fee for new organisations while grandfathering the first cohort for three extra months. Pilot the Partner tier with five agencies that already budget for student collaborations. Success metrics: project completion rate above 85\% and churn below 10\% per quarter.
  \item \textbf{Mature growth (beyond month 12).} Scale the fee to everyone, roll out the Partner tier broadly, and launch the insights add-on for universities. At this stage we track revenue per active organisation, NPS, and uptake of enablement tools to ensure monetisation deepens engagement rather than cannibalising it.
\end{enumerate}

\subsection*{Experiments and guardrails}
To keep ourselves honest we line up three experiments. First, an A/B test on when organisations see the completion fee: at project submission or only after matching. We monitor match acceptance and dropout rates. Second, pricing interviews using the Van Westendorp method to validate the Partner tier before we hard-code the 499 DKK price point; this leans on \citet{Reillier2017}'s advice to co-design producer tools. Third, a diary study with 15 students to learn whether insights reporting feels empowering or creepy, so we can adjust the consent flows. Each experiment has a rollback plan documented in Notion. Writing it out in English keeps the tone scrappy and reflective while answering the exam's monetisation brief directly.
