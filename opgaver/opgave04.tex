\section*{Assignment 04}
\addcontentsline{toc}{section}{Assignment 04}

In this section I map the most realistic ways we can monetise the platform without suffocating the fragile network effect we are still trying to ignite.\footnote{Quick recap: the platform matches local home-chef hosts with curious guests who want intimate dining experiences.} The logic follows classic platform thinking: give at least one side a free ride early on to spark adoption \citep{Choudary2016}. Translating the original notes into English and doubling the exposition lets me unpack the strategic logic instead of tossing out bullet-point guesses.

\subsection*{Potential revenue streams}
\begin{itemize}
  \item \textbf{Transaction fee on each match.} We could take, say, 8\% of the payment, but only after both host and guest confirm satisfaction. The model scales cleanly and sits close to the value we already orchestrate \citep{HagiuWright2013}. The risk is that early power users may feel punished and jump to informal channels if we crank the fee too high or too fast \citep{Reillier2017}. To soften that reaction, we would roll out the fee with transparent messaging, offer fee holidays for hosts who fill niche cuisines, and monitor elasticity through cohort analysis.
  \item \textbf{Freemium membership for hosts.} Everyone keeps a free baseline, while a paid Pro tier unlocks faster payouts, better visibility, and auto-generated shopping lists. The upside is predictable MRR and the ability to empower semi-professional hosts who crave better tooling \citep{Choudary2016}. The downside is the optics of pay-to-win dynamics that skew the community toward pros. We mitigate that by ensuring host badges reflect guest satisfaction, not subscription status, and by offering scholarships to community organisers who run inclusive events.
  \item \textbf{Advertising or data insights for local food brands.} \citet{Srnicek2017} would call this the classic datafication play: harness behavioural data to sell targeted campaigns. It could become a monster revenue stream later, but \citet{Zuboff2019} reminds us surveillance capitalism erodes trust fast. For an intimate food community it feels off-brand, so we keep it on the long-term ``maybe'' list while investing in privacy-first experimentation, like aggregated demand heatmaps that never expose individual behaviour.
  \item \textbf{Experience bundles with travel partners.} Doubling the narrative gives me room to add a fourth option we only hinted at before: partnering with boutique hotels or travel agencies to offer weekend packages that include a hosted dinner. This aligns with \citet{Reillier2017}'s idea of layering complementary services once the core interaction is stable. Revenue would arrive via referral fees, but we would pilot it carefully to avoid diluting the neighbourhood vibe that makes the platform unique.
\end{itemize}

\subsection*{Choice and timing across the lifecycle}
I advocate a sequential model: secure critical mass, then gradually dial up monetisation so we follow the classic seeding $\rightarrow$ growth $\rightarrow$ harvest pattern \citep{Choudary2016}.
\begin{enumerate}
  \item \textbf{Seeding phase (0--6 months post-beta).} No fees, no subscriptions. The focus is on hitting 500+ successful dinners and collecting rich reviews. Behind the scenes we test the payment infrastructure with dummy transactions so we know it works, but we do not collect revenue yet. We also gather qualitative feedback on willingness to pay, building a dataset that informs later pricing experiments.
  \item \textbf{Early growth (months 7--12).} We introduce a soft 4\% transaction fee for new hosts while existing hosts keep zero fees for three extra months. That allows us to measure elasticity without shocking the core side. In parallel we launch a voluntary Pro subscription at 149 DKK/month targeted at hosts running two or more events monthly. We bundle in community perks---recipe swaps, mentoring circles---to keep the tone collaborative rather than extractive.
  \item \textbf{Mature growth (post month 12).} Once retention stabilises and NPS stays above 50, we scale the fee to 8\% for everyone and stack add-ons inside the Pro plan (like partnerships with local retailers or insurance coverage). At this stage we still avoid ads and data sales, but we keep scenario plans on the shelf in case we cross 50k monthly users and can design privacy-first formats. The expanded write-up lets me spell out the exact metrics---repeat bookings, host churn, guest lifetime value---that signal when to flip each switch.
\end{enumerate}

\subsection*{Planned experiments}
To avoid guesswork I schedule three concrete experiments and now give them richer context:
\begin{itemize}
  \item \textbf{A/B test for fee onboarding.} Half of new hosts encounter the 4\% fee after their third event, while the control group sees it after the fifth. We track churn, average event frequency, and qualitative sentiment from exit interviews. The extra explanation clarifies how we define success and what fallback plan (rolling back fees for specific cuisines) we keep in reserve.
  \item \textbf{Price-ladder test for Pro.} We rotate three price points (129, 149, 179 DKK) and monitor upgrade rate and CLV. It is a classic willingness-to-pay experiment straight out of \citet{ShapiroVarian1999}. With more words I can show how we balance quantitative results with fairness concerns: if higher prices crowd out community hosts, we cap the tier and open a ``sponsor a host'' fund.
  \item \textbf{Qualitative diary study on advertising.} Before entertaining brand collaborations, we recruit 15 users for a two-week diary study capturing how sponsored content affects their experience. That gives us evidence if we later push back against aggressive data models when talking to investors, and it keeps us accountable to \citet{Zuboff2019}'s critique rather than hand-waving it away.
\end{itemize}

Overall we keep core monetisation tightly coupled to the value we already facilitate (matching hosts and guests) and postpone the more invasive models until we have both scale and trust capital. The English translation plus extra commentary turn a short list into a defensible strategy with explicit guardrails.
