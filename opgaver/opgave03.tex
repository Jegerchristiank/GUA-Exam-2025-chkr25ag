\section*{Opgave 03}
\addcontentsline{toc}{section}{Opgave 03}

\subsection*{Mekanismer på studiesiden}
For studerende har vi allerede et par konkrete praksisgreb i spil. Først og fremmest kører vi universitetsmail-verifikation og et obligatorisk profilsetup, så der er et minimum af tillid, før nogen matcher op \citep{Choudary2016}. Dernæst har vi lagt en simpel projektlog ind, hvor studerende afleverer korte statusopdateringer og evalueringer, så vi kan måle, om kerneinteraktionen faktisk giver læring og ikke bare ekstra arbejde \citep{Reillier2017}. Kombinationen hjælper os med at holde fast i Hagiu \& Wrights pointe om at beskytte kerne-transaktionen fra støjende sideaktiviteter \citep{HagiuWright2013}.

På tegnebrættet ligger der to udvidelser. Vi vil bygge en ``skill passport''-feature, som automatisk samler feedback og badges fra afsluttede projekter. Det matcher både \citet{Choudary2016}'s ramme om værdifangst gennem data og \citet{ShapiroVarian1999}'s idé om at øge switching costs ved at gøre historikken værdifuld. Dernæst planlægger vi et letvægts mentorforløb, hvor alumni kan give sparring på igangværende cases. Det er inspireret af \citet{Reillier2017}, der taler for at orkestratoren løfter kvaliteten gennem ``curation'' fremfor at eje selve læringen.

\subsection*{Mekanismer på NGO-siden}
På efterspørgselssiden har vi allerede en forhåndsscreening af NGO-profiler og et template-bibliotek til projektbriefs. Screeningen følger \citet{FarrellSaloner1986}'s logik om kompatibilitet: vi vil undgå projekter, der kræver proprietære systemer, som skræmmer studerende væk. Templates reducerer skrivearbejdet og gør matchingen hurtigere, hvilket holder os i tråd med minimum viable participation-tankegangen fra \citet{Reillier2017}. Vi har også sat en let reputationsscore på NGO-profiler, hvor afsluttede projekter giver en lille ``impact''-badge; det er vores måde at operationalisere \citet{Choudary2016}'s fokus på tillidsmekanismer i platform-governance.

Fremadrettet vil vi gerne tilbyde en data-light effektrapport, der visualiserer studerendes output med NGO'ernes egne KPI'er. Det skal ske med fuldt samtykke og begrænset datalagring, så vi undgår at glide over i overvågningskapitalismens mørke afkroge, som \citet{Zuboff2019} advarer om. Derudover arbejder vi på et abonnementsbaseret supportlag, hvor NGO'er kan få adgang til workshops om ``how to scope a student project''. Det bygger på \citet{HagiuWright2013}'s observation om, at platforme må investere i værktøjer, der sænker transaktionsomkostningerne for den mest ressourcetrængte side.

\subsection*{Samlet kobling til teori}
Pointen i hele setup'et er, at praksisgrebene ikke står alene. Vi prøver at få krydsside-netværkseffekterne til at trives ved at beskytte den centrale interaktion (studie\,$\leftrightarrow$\,NGO) med governance, der føles let, men faktisk er ret struktureret \citep{Choudary2016}. Samtidig holder vi øje med, at hver ny feature leverer en målbar forbedring i ``minimum viable participation'', sådan som \citet{Reillier2017} foreslår. På den måde balancerer vi kortsigtet adoption med langsigtet differentiering, mens vi hele tiden spørger, om vi tilfører reel værdi til begge sider fremfor bare at bygge pynt.
