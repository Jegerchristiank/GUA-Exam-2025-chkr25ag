\section*{Assignment 03}
\addcontentsline{toc}{section}{Assignment 03}

\subsection*{Mechanisms on the student side}
For students we already operate with a handful of tangible governance levers. First we require university-email verification and a mandatory profile setup so there is at least a baseline of trust before anybody matches up \citep{Choudary2016}. Next we added a lightweight project log where students submit short status updates and reflections; that lets us track whether the core interaction delivers learning rather than just unpaid labour \citep{Reillier2017}. Translating the original Danish notes into English also gave me space to justify why each lever matters: the verification reduces fake accounts, the log creates a shared memory, and together they keep us faithful to \citet{HagiuWright2013}'s warning about protecting the core transaction from noisy side quests.

With the doubled word count I can map the next two features on the roadmap. We plan to launch a ``skill passport'' that aggregates feedback and badges from completed projects, which lines up nicely with \citet{Choudary2016}'s data-value capture and \citet{ShapiroVarian1999}'s idea of raising switching costs by making history valuable. We also outline a lightweight mentorship track where alumni volunteer guidance on live cases. It is inspired by \citet{Reillier2017}, who emphasise that orchestrators can lift quality via careful curation rather than by owning the learning itself. The extra detail clarifies how we keep the student experience aspirational without drifting into extractive territory.

\subsection*{Mechanisms on the NGO side}
On the demand side we currently run a pre-screening flow for NGO profiles and provide a template library for scoping projects. The screening follows \citet{FarrellSaloner1986}'s compatibility logic: we want to filter out projects that rely on proprietary systems or exploitative expectations that scare students away. Templates cut writing time and streamline matching, keeping us aligned with the minimum-viable-participation mindset from \citet{Reillier2017}. We also added a light reputation score where completed projects yield an ``impact'' badge; that is our practical take on \citet{Choudary2016}'s governance focus on trust cues.

Looking ahead, we want to offer a data-light impact report that visualises student contributions alongside each NGO's internal KPIs. That needs to happen with explicit consent and tight data retention so we do not wander into the surveillance capitalism shadows that \citet{Zuboff2019} warns about. In parallel we are prototyping a subscription-based support tier where NGOs get access to ``how to scope a student project'' workshops. The idea draws on \citet{HagiuWright2013}'s observation that platforms often have to invest in tools that lower transaction costs for their most resource-strapped side. Doubling the length gave me room to explain the operational steps---from onboarding scripts to follow-up surveys---that make these mechanisms credible instead of wishful thinking.

\subsection*{Bringing the theory together}
Across both sides the governing idea is that none of these tactics stand alone. We are trying to cultivate cross-side network effects by protecting the central student\,$\leftrightarrow$\,NGO interaction with governance that feels easy but is actually quite structured \citep{Choudary2016}. Each new feature must prove it improves minimum viable participation in line with \citet{Reillier2017}. By narrating the logic in English and stretching it to twice the original length, I can surface the trade-offs, list the metrics we track, and show how we avoid adding shiny distractions. The result is a clearer link between theory and practice, with just enough informal tone to remind readers this is still a student project learning in public.
