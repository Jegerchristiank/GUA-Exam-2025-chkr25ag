\section*{Assignment 03: Evolution of the Platform Concept}
\addcontentsline{toc}{section}{Assignment 03: Evolution of the Platform Concept}

\subsection*{Where we started}
Our very first sketches revolved around a ``dinner experiences'' marketplace that matched home chefs with curious guests. It fit the zeitgeist but never quite aligned with why we enrolled in the course. Early interviews with classmates, plus the VirtuAI quick-case debrief \citep{Gunasilan2024}, exposed two red flags: regulators already scrutinise informal food businesses, and our team had zero advantage in logistics. When we overlaid \citet{Choudary2016}'s typology, we realised we were drifting toward an asset-heavy service, not the lightweight orchestrator we wanted to study.

\subsection*{Moments that changed the trajectory}
The pivotal moment came during Session~6 when a guest NGO described how hard it is to scope student projects without hand-holding. That story made us revisit our own campus experience and birthed SkillSync: a student--organisation matchmaking platform focused on scoped, time-bound collaborations. We mapped the new interaction using the platform design toolkit from \citet{Reillier2017}, prototyped scoping templates in Figma, and ran hallway tests with five NGOs from previous course projects. Another turning point was analysing monetisation for the home-chef idea. The numbers crumbled under \citet{Porter2008}'s competitive pressure, yet the same analytical exercise illuminated how SkillSync could monetise through completion-based fees and partner enablement. The pivot looked dramatic on paper, but in practice it was a sequence of incremental bets guided by data and theory.

\subsection*{Reflection on the path taken}
Was sticking with SkillSync the optimal play? Mostly yes. The concept aligns with our comparative advantage (campus networks, experience with student consulting) and gives us a clean cross-side interaction to analyse. Still, we moved too slowly on validating organisational willingness to pay. If I could rewind, I would run pricing conversations in parallel with prototyping instead of waiting for a polished deck---\citet{HagiuWright2013} warn that deferring business-model validation makes pivots harder later. I would also keep a thinner backlog so we spot sunk-cost bias earlier; the team clung to unused artefacts from the food-marketplace experiment because we had invested in them. Writing this reflection in English let me document the messy middle, acknowledge the road not taken, and show the learning loops that question three explicitly asks for.
