\section*{Opgave 10}
\addcontentsline{toc}{section}{Opgave 10}

Det sidste jeg mangler er at folde blikket ud mod de næste fem år og være ærlig om både drømme og faldgruber. Så her kommer mi
n fremskrivning, et reality-check på trusler og nogle afsluttende refleksioner over hele platformrejsen.

\subsection*{Fremskrivninger}
Jeg har sat et konservativt basisscenarie op med tre nøgletal, hvor vi bygger oven på alt det groundwork fra de tidligere opga
ver:\newline
\begin{table}[h]
  \centering
  \begin{tabular}{p{3cm}p{3.5cm}p{6cm}}
    \toprule
    \textbf{År 5-mål} & \textbf{Tal} & \textbf{Antagelser} \\
    \midrule
    Aktive brugere & 75.000 & Årlig vækst på 55\% drevet af lokale cluster-launches og netværkseffekter, fastholdelse på 68\% \citep{Choudary2016,Srnicek2017}. \\
    Omsætning & 42 mio. DKK & Hybridmodel: 60\% transaktionsgebyr (4\%), 25\% datadrevne abonnementer, 15\% co-branded partnerskaber \citep{ShapiroVarian1999}. \\
    Strategiske partnerskaber & 18 & Tre nationale ankerorganisationer, fem branchedata-hubs, ti kommunale eller regionale innovationsenheder \citep{Reillier2017}. \\
    \bottomrule
  \end{tabular}
  \caption{Basisscenarie for platformens udvikling over fem år.}
\end{table}

Tallene bygger på de pilotdata vi allerede har (konvertering ca. 12\%) og at vi i år 2-3 får automatiseret onboarding, så partn
erer kan integrere sig uden specialudvikling. Skulle vi se stærkere virale loops, har jeg et upside-scenarie i baghånden, hvor
 både brugere og omsætning ligger 30\% højere, men det kræver, at community-featuresne faktisk rammer plet.

\subsection*{Trusler og exit-scenarier}
De største trusler er stadig klassikerne fra platformlitteraturen: Substitution, multi-homing og regulering. Hvis en global ak
tør køber sig ind i segmentet og dumper priserne, kan vores transaktionsgebyrer pludselig se dyre ud \citep{Porter2008}. Og hvi
s vi ikke bliver ved med at gøre datahåndteringen gennemsigtig, kan både partnere og myndigheder lukke ned for datastrømmene, h
vilket effektivt kvæler netværkseffekterne \citep{Srnicek2017}. For at modvirke multi-homing skal vi blive ved med at gøre smar
te integrationsfeatures eksklusive og dyrke switching costs gennem historiske insights, som konkurrenter ikke kan kopiere uden
 videre \citep{FarrellSaloner1986}.

Jeg har skitseret to realistiske exit-scenarier, hvis alt går galt: (1) et kontrolleret acqui-hire, hvor en større nordisk bran
cheplatform køber team og IP, mens vi lukker selve markedspladsen ned under ordnede forhold; (2) en pivot til ren datainfrastru
ktur, hvor vi afvikler matchmakingen men viderefører API-laget som SaaS mod en mindre kundegruppe. Begge scenarier kræver, at v
i holder koden modulær og kontrakterne rene, så værdierne kan løsrives uden kaos \citep{Reillier2017}.

\subsection*{Afsluttende refleksion}
Det her forløb har været en reminder om, hvor krævende det er at balancere vækstambitioner med governance. Hver eneste designbe
slutning har ramt begge sider af markedspladsen på én gang, og det har tvunget mig til at tænke i loops frem for lineære funnels
 \citep{Choudary2016}. Samtidig har Porters konkurrencestrategi stadig været en god virkelighedscheck på, om vi faktisk er diff
erentierede eller bare endnu en generisk SaaS \citep{Porter2008}. Den store læring er, at netværkseffekter ikke er en gratis sn
arvej: de opstår kun, hvis vi hele tiden investerer i legitimitet, datakvalitet og partnerskaber, der giver mening for begge sid
er. Det er præcis derfor, jeg afslutter med et fokus på partnerskabsprogrammet og en exit-plan---for når governance er på plads
, kan vi både vokse og trække stikket på en ordentlig måde \citep{Srnicek2017}.
