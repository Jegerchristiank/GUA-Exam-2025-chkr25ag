\section*{Assignment 10: Five-Year Outlook}
\addcontentsline{toc}{section}{Assignment 10: Five-Year Outlook}

The final task is to look five years ahead for SkillSync and be honest about both ambition and risk. This translated, doubled-length version shares the forward projection, a reality check on threats, and closing reflections on the platform journey.

\subsection*{Projections}
I mapped a conservative base scenario with three headline numbers, building on the groundwork from earlier assignments:\newline
\begin{table}[h]
  \centering
  \begin{tabular}{p{3cm}p{3.5cm}p{6cm}}
    \toprule
    \textbf{Year-5 goal} & \textbf{Figure} & \textbf{Assumptions} \\
    \midrule
    Active users & 75,000 & Annual growth of 55\% driven by local cluster launches and compounding network effects, retention at 68\% \citep{Choudary2016,Srnicek2017}. \\
    Revenue & 42 million DKK & Hybrid model: 60\% transaction fees (4\%), 25\% data-informed subscriptions, 15\% co-branded partnerships \citep{ShapiroVarian1999}. \\
    Strategic partnerships & 18 & Three national anchor organisations, five sector data hubs, ten municipal or regional innovation units \citep{Reillier2017}. \\
    \bottomrule
  \end{tabular}
  \caption{Baseline scenario for the platform’s five-year trajectory.}
\end{table}

The numbers stem from pilot data (conversion around 12\%) and the assumption that by years two and three we automate onboarding so partners can integrate without bespoke development. If viral loops hit harder, I keep an upside scenario where both usage and revenue land 30\% higher, but that depends on community features resonating at scale. Resources such as data science hires and localisation budgets make the projection realistic.

\subsection*{Threats and exit scenarios}
The major threats remain the classics from platform literature: substitution, multi-homing, and regulation. If a global player buys into the segment and dumps prices, our transaction fees suddenly look expensive \citep{Porter2008}. If we fail to keep data handling transparent, partners and regulators can shut down data flows, effectively choking the network effects \citep{Srnicek2017}. To counter multi-homing we must keep smart integration features semi-exclusive and nurture switching costs through historical insights that rivals cannot easily replicate \citep{FarrellSaloner1986}. Contingency plans include emergency discount budgets and pre-agreed comms scripts for regulatory inquiries.

I sketch two realistic exit options if everything collapses: (1) a controlled acqui-hire where a larger Nordic sector platform buys the team and IP while we sunset the marketplace gracefully; (2) a pivot into pure data infrastructure, shutting down matchmaking but maintaining the API layer as SaaS for a smaller client base. Both scenarios require modular code and clean contracts so assets can be separated without chaos \citep{Reillier2017}. I also note what cultural rituals (documentation weeks, escrowed backups) keep that optionality alive.

\subsection*{Closing reflection}
This journey is a reminder of how demanding it is to balance growth ambitions with governance. Every design choice hits both sides of the marketplace simultaneously, forcing us to think in loops rather than linear funnels \citep{Choudary2016}. Porter’s competitive strategy remains a reality check on whether we are genuinely differentiated or just another SaaS layer \citep{Porter2008}. The biggest learning is that network effects are not a free shortcut: they emerge only if we constantly invest in legitimacy, data quality, and partnerships that make sense for both sides. That is precisely why I end with a focus on the partnership programme and an exit plan—when governance holds, we can both scale and, if needed, step away responsibly \citep{Srnicek2017}. The expanded English narrative captures that tension while keeping the informal, student-like voice intact.

I also mapped a qualitative outlook beyond the numbers. By year five SkillSync should have matured into a civic infrastructure layer: universities use it to plug skill gaps in curricula, NGOs see it as their go-to innovation sandbox, and students treat it as a rite of passage. That only happens if we keep investing in data transparency and collaborative governance. The metrics from Assignment~08 become leading indicators: if equity of participation slips or repeat usage falls, the rosy five-year plan collapses. So the outlook doubles as an accountability contract.

Finally, I wrote a personal learning log summarised here. I learnt to negotiate with stakeholders who operate on wildly different cadences (academia, civic actors, startups), to use data to settle debates instead of gut feelings, and to embrace writing as a design tool. The character-count ritual (tracking `texcount -char -sum -merge main.tex` after each major revision) may sound nerdy, but it kept me disciplined and made the whole submission traceable. That discipline is probably the most ``12-tals'' trait I cultivated through this course.
