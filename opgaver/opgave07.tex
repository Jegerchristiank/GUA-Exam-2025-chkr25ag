\section*{Opgave 07}
\addcontentsline{toc}{section}{Opgave 07}

Jeg starter med at kortlægge, hvem der egentlig falder igennem vores platform i dag. NGO'er med små budgetter er tydelige, fordi de mangler både penge og folk til at babysitte endnu et dashboard. Mange kommer fra frivilligt drevne civilsamfundsprojekter, så deres digitale modenhed er lav, og de frygter at hænge på skjulte gebyrer eller dataforpligtelser, som \citet{Srnicek2017} advarer mod i sin kritik af platformkapitalisme. Dernæst har vi de forskellige fakulteter på campus, især de humanistiske og designfaglige miljøer, som arbejder med andre succeskriterier end business-folkene. Hvor handelshøjskolens institutter jagter KPI'er, er kunst- og kulturstudier mere optaget af deltagelse og fælles ejerskab, hvilket passer med \citet{Choudary2016}'s pointe om, at platforme skal tune governance til segmentets værdilogik.

For at gøre hverdagen lettere for NGO'erne foreslår jeg et ``lean onboarding kit'': et færdigt datasheet, skabeloner til events og et adgangsprogram, hvor vi parrer dem med studerende, som kan hjælpe i de første uger. I VirtuAI-casen blev den sociale onboarding helt afgørende for at få non-profits ombord, netop fordi de havde få ressourcer \citep{Gunasilan2024}. Teknisk betyder det, at platformens standardflow får en light-version med klare budgetlofter og automatisk genererede rapporter, så organisationerne slipper for at bygge deres egne måleværktøjer.

Når jeg kigger på fakulteterne, handler designgrebet om at lade dem forme deres egne mikrofællesskaber. Vi kan oprette ``faculty sandboxes'', hvor humaniora kan definere alternative engagement-metrics, mens økonomerne kan holde fast i klassiske vækstkurver. Det matcher \citet{Reillier2017}'s råd om modulære governance-lag, der ikke låser fællesskaberne fast i én logik. Samtidig skal vi være åbne for at nogle fakulteter vil eksperimentere med analoge events, som kan dokumenteres gennem simple upload-formularer i stedet for obligatorisk live-streaming.

På policy-siden ville jeg skrive tre simple regler. Først en fairness-paragraf, der forpligter os til at måle ressourceforbrug pr. organisation og tilbyde fee-waivers, hvis andelen af frivilligt arbejde overstiger et vist niveau. Det spiller sammen med \citet{ShapiroVarian1999}, som peger på, at subsidier til den svage side kan accelerere netværkseffekter. Dernæst en inklusionspolitik, hvor hvert fakultet får sæde i et data- og etikråd, så vi undgår bias i governance---noget \citet{Zuboff2019} ellers viser som en klassisk fælde i overvågningskapitalismen. Endelig en løbende effekt-audit inspireret af DineTogether-casen, hvor vi hver kvartal gennemgår, om funktioner utilsigtet favoriserer ressource-stærke aktører \citep{Rennella2023}.

Som samlet designprincip holder jeg fast i ``progressive engagement'': Jo mere ressourcestærk en aktør er, desto flere avancerede værktøjer låser vi op, mens basisoplevelsen er super simpel og gratis. Det er en måde at operationalisere både teoriens krav om balancerede netværkseffekter og praksis fra cases, så NGO'er med få midler og fakulteter med andre succeskriterier føler sig set og stadig kan byde ind uden at blive overrumplet.
