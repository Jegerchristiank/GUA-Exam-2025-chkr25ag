\section*{Assignment 07}
\addcontentsline{toc}{section}{Assignment 07}

I begin by mapping who falls through the cracks on our platform today. Resource-light NGOs are the obvious case: they lack cash and staff to babysit yet another dashboard. Many operate volunteer-driven civil-society projects, so their digital maturity is low and they fear hidden fees or data obligations of the kind \citet{Srnicek2017} warns about in his critique of platform capitalism. Next come the different faculties on campus, especially humanities and design departments that live by success metrics wildly different from the business school crowd. Commerce institutes chase KPIs, whereas arts and culture studies prize participation and shared ownership, echoing \citet{Choudary2016}'s point that governance must align with each segment’s value logic.

To make life easier for NGOs I propose a ``lean onboarding kit'': a ready-made data sheet, templated event briefs, and an access programme where we pair them with students who can help during the first weeks. In the VirtuAI case the social onboarding layer was crucial for getting nonprofits involved precisely because their resources were thin \citep{Gunasilan2024}. Technically this means the standard flow gets a lightweight variant with clear budget caps and auto-generated reports so organisations skip building measurement tools from scratch. The expanded English version also lets me spell out support rituals (office hours, template libraries, Slack buddies) that were only implicit before.

When I look at faculties, the design move is to let them shape their own micro-communities. We can spin up ``faculty sandboxes'' where humanities define alternative engagement metrics while economists stick with classic growth curves. That mirrors \citet{Reillier2017}'s advice on modular governance layers that avoid locking communities into a single logic. We should also stay open to certain faculties experimenting with analogue events that can be documented via simple upload forms rather than mandatory livestreaming. Doubling the content makes room to explain how we staff these sandboxes and keep knowledge flowing between them.

On the policy side I sketch three straightforward rules. First a fairness clause committing us to track resource spend per organisation and offer fee waivers if volunteer hours exceed a certain threshold. That dovetails with \citet{ShapiroVarian1999}, who note that subsidising the weaker side can accelerate network effects. Second an inclusion policy granting every faculty a seat on a data-and-ethics council so we avoid governance bias---something \citet{Zuboff2019} flags as a classic trap in surveillance capitalism. Third a recurring impact audit inspired by the DineTogether case, where every quarter we review whether features inadvertently favour resource-rich actors \citep{Rennella2023}.

As an overarching design principle I stick with ``progressive engagement'': the more resources an actor has, the more advanced tools we unlock, while the baseline experience stays super simple and free. It is a practical way to operationalise both the theoretical demand for balanced network effects and the pragmatic lessons from our cases. The doubled length makes it clear how NGOs with minimal budgets and faculties with divergent success criteria can still join without feeling overwhelmed, while ambitious partners still see a path to deeper collaboration.
