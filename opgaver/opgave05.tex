\section*{Opgave 05}
\addcontentsline{toc}{section}{Opgave 05}

\subsection*{Onboarding, feedbacksløjfer og moderation}
Jeg ser onboarding som en blanding af storytelling og friktionstest: først møder folk en klar, jordnær value prop i landing flowet, derefter inviterer vi dem ind via en guided tour, der viser de vigtigste interaktioner (fx hvordan man poster, booker eller matcher). Det ligger direkte i forlængelse af platform design-tankegangen om at hjælpe brugere hurtigt frem til deres første successfulde transaktion, ellers dør netværkseffekten i fødslen \citet{Choudary2016}. Derfor deler jeg onboarding op i et par faser: (1) pre-signup nudges (kort video og social proof), (2) profil-setup med præ-udfyldte forslag, og (3) et "first mission" checklist, der giver badges når de har prøvet kernen. Samtidig matcher vi nye brugere med mentorer eller automatiske prompts, så der altid er et svar i indbakken inden for den første time. Feedbacksløjferne er bygget ind i flowet: efter hver kernehandling spørger vi til oplevelsen (1-klik rating og fritekst), vi monitorerer feature adoption via cohort-dashboards, og vi sender ugentlige opsummeringer til creators/moderatorer med deres performance. Det er sådan vi holder øje med positive recency-effekter og kan tweake priser eller regler, så begge sider stadig får værdi \citet{Reillier2017}. Moderationsprocessen kører i tre lag: automatisk filtrering (keyword detection og adfærdsflags), community moderation (trusted users kan skjule indhold midlertidigt), og til sidst et professionelt team, der vurderer klager inden for 24 timer. Hele setuppet er tænkt som en løbende læringssløjfe: når teamet ændrer regler, oversætter vi dem med det samme til onboarding-materialet og sender push-nyheder, så folk føler sig taget i hånden i stedet for overrumplet.

\subsection*{Datapolitikker og etik}
Dataindsamling sker efter minimalt-princippet: vi tager kun det, der er nødvendigt til at drive matching og tillidsmekanismer (profiloplysninger, transaktionshistorik, kvalitets-feedback). Platformlogikken frister til mere, men vi har lært af surveillance-kritikken, at overdreven indsamling underminerer legitimitet \citet{Zuboff2019}. I brugen af data har vi et klart hierarki: først serviceforbedring (recommender justeringer, svindelkontrol), derefter ansvarlig personalisering (ikke manipulativ nudging), og kun i tredje omgang kommercielle analyser for partnere. Vi bruger differential privacy i aggregerede rapporter, så enkeltindivider ikke kan genkendes, og vi indbygger fairness-kontrol i algoritmerne for at spotte bias, inspireret af debatten om platform kapitalisme og magtubalancer \citet{Srnicek2017}. Transparens betyder, at vi serverer en "dataspejl"-side: alle brugere kan se hvilke datapunkter vi har, hvorfor vi har dem, hvor længe de gemmes, og hvordan man kan redigere eller slette dem. Derudover udsender vi kvartalsvise ansvarlighedsrapporter, hvor vi deler statistik over moderation, databrud (hvis nogen), og hvilke ændringer vi har lavet i algoritmiske beslutningssystemer. Etikken er ikke bare compliance; vi kører et internt ethics review board, hvor produktteams skal pitche nye eksperimenter og vise, at de ikke skubber power ubalancer, som platformøkonomien ellers er notorisk for \citet{Choudary2016}. Kombineret gør det, at datapolitikken både kan tåle regulatorisk spotlight og stadig føles fair for vores community.
